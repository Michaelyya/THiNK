\documentclass[12pt,a4paper]{report}
\usepackage[utf8]{inputenc}
\usepackage{amsmath}
\usepackage{amsfonts}
\usepackage{amssymb}

\title{Massachusetts Institute of Technology: Physics Department 8.044 Statistical Physics I Spring Term 2013}
\date{Due: 12:40 PM, Wednesday, April 10}

\begin{document}

\maketitle

\section*{Problem Set \#8}

\section*{Problem 1: Dust Grains in Space}

Astronomers have discovered that there exist in the interstellar medium clouds of “dust grains” whose chemical composition may include silicates (like sand) or carbon-containing compounds (like graphite or silicon carbide). Evidence indicates that many of these grains have a needle like shape.

\begin{align*}
\label{eq:Hamiltionan}\tag{1}
H &= \frac{L_1^2}{2I_1} + \frac{L_2^2}{2I_2} + \frac{L_3^2}{2I_3}
\end{align*}

Assume that a dust cloud is in thermal equilibrium at a temperature $T$ .

\section*{Problem 2: Adsorption On a Stepped Surface}

Consider $N$ identical xenon atoms adsorbed on a silicon surface which has a total of $M$ possible adsorption sites.

\section*{Problem 3: Neutral Atom Trap}

A gas of $N$ indistinguishable classical non-interacting atoms is held in a neutral atom trap by a potential of the form $V (r) = ar^{2}^{1/2}$ where $r = (x^2 + y^2 + z^2)$

\section*{Problem 4: Two-Dimensional H2 Gas}

\section*{Problem 5: Why Stars Shine}

\end{document}