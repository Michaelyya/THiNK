
\documentclass{article}
\usepackage{authblk}
\usepackage{amsmath}
\title{18.01 Calculus}
\author[1]{Jason Starr}
\affil[1]{Massachusetts Institute of Technology}
\date{Fall 2005}
\begin{document}
\maketitle
\begin{abstract}
Problem Set 2 due on Sept. 30, 2005, by 2:00pm sharp. This problem set consists of two parts: Part I and Part II. Here are the solutions.
\end{abstract}
\section*{Part I: (20 points)}
\subsection*{(a) (2 points) p. 97, Section 3.3, Problem 44}
Solution: Let $u = 8 - x^{2}$, and let $v = u$. Then $y = \frac{x}{v}$. By the quotient rule, $y' = \frac{2}{v^{3}} ((x) v' - x(v)) = \frac{2}{v^{2}} (v - x(v'))$. By the chain rule, $v' = \frac{dv}{dx} = \frac{du}{dx} = (5u^{4})(-2x)$. When $x$ equals $3$, $u$ equals $8 - (3)^{2} = -1$ and $v$ equals $(-1)^{5} = -1$. Thus $v'(3)$ equals $5.\frac{((-1)^{4})(-2 \cdot 3)}{1} = -30$. And, $y'(3) = \frac{1}{(-1)^{2}} ((-1) - (3)(-30)) = -1 + 90 = 89$. Therefore, the slope of the tangent line at $(3, -3)$ is, $y = 89(x - 3) + (-3)$, $y = \frac{89x - 270}{1}$.

\subsection*{(b) (2 points) p. 107, Section 3.5, Problem 15}
Solution: Implicit differentiation gives $\frac{d \left(1 - \frac{y}{1 + y}\right)}{dx} = 1$. By the chain rule, $\frac{d \left(1 - \frac{y}{1 + y}\right)}{dx} = \frac{d \left(\frac{1}{1 + y} - y\right)}{dx}$. By the quotient rule, $\frac{dy}{dx} \left(\frac{1 - y}{1 + y}\right) = \frac{(1 + y)^{2}(-1) - (1 - y)}{(1 + y)^{2}} = -\frac{(1 + y)^{2}}{2}$.\\
Thus, implicit differentiation gives $\frac{dy}{dx} = -\frac{(1 + y)^{2}}{2}$.\\
To solve for $x$, multiply both sides of the equation by $1 + y$ to get, $1 - y = x(1 + y) = x + xy$. Add $y - x$ to each side of the equation to get, $1 - x = xy + y = (x + 1)y$. Divide each side of the equation to get, $y = \frac{1 - x}{1 + x}$. By the quotient rule, $y' = \frac{(1 + x)^{2}}{2} = (1 - x) - (1 + x)$.\\
Since $y = \frac{1 - x}{1 + x}$, $1 + y$ equals $\frac{1 + x + 1 - x}{1 + x} = \frac{2}{1 + x}$. Thus, $y' = -\frac{(1 + y)^{2}}{2} = -\frac{1}{2} = -\frac{1}{2} \cdot \frac{4}{(1 + x)^{2}}$. Therefore, $y' = -\frac{(1 + y)^{2}}{2} = -\frac{1}{2} \cdot \frac{(1 + x)^{2}}{\frac{1}{2}} = -\frac{1}{2} \cdot \frac{1}{(1 + x)^{2}}$.

% The same process should be done for the rest of the problems.

\section*{Part II: (30 points)}
\subsection*{Problem 1: (5 points)}
Find the equation of the tangent line to the graph of $y = e^{571x}$ containing the point $(102 \pi, 0)$. This is not a point on the graph; it is a point on the tangent line.
\subsubsection*{Solution to Problem 1}
Denote $571$ by the symbol $n$. Denote $102 \pi$ by the symbol $a$. The derivative of $e^{nx}$ equals $n e^{nx}$. Thus the slope of the tangent line to $y = e^{nx}$ at the point $(b, e^{nb})$ equals $n e^{nb}$. So the equation of the tangent line to $y = e^{nx}$ at $(b, e^{nb})$ is, $y = n e^{nb}(x - b) + e^{nb} = n e^{nb}x - (nb - 1)e^{nb}$.\\
If $(a, 0)$ is contained in this line, then the equation holds for $x = a$ and $y = 0$, $0 = n e^{nb}a - (nb - 1)e^{nb} = (na - nb + 1)e^{nb}$. Since $e^{nb}$ is not zero, dividing by $e^{nb}$ gives, $na - nb + 1 = 0$, from which we get $b = \frac{na + 1}{n}$. Substituting this in gives the equation of the tangent line to $y = e^{nx}$ containing $(a, 0)$, $y = nax - (na + 1)$.

\subsection*{Problem 2: (5 points)}
(a) (2 points) What does the chain rule say if $y = x^{a}$ and $u = y^{b}$? The constants $a$ and $b$ are fractions.\\
Solution: First of all, $u$ equals $y^{b}$, which equals $(x^{a})^{b}$. By the rules for exponents, this equals $x^{ab}$. According to the chain rule, $\frac{du}{dx} = \frac{du}{dy} \cdot \frac{dy}{dx} = (by^{b-1})(ax^{a-1})$. Substituting in $y = x^{a}$ gives, $\frac{du}{dx} = (b(x^{a})^{b-1})(ax^{a-1})$. Using the rules for exponents, this equals, $\frac{du}{dx} = (b_{x}^{ab-b})(ax^{a-1}) = abx^{ab-1}$. This is precisely what the chain rule should give, since, setting $c = ab$, $\frac{d(x^{c})}{dx} = cx^{c-1} = abx^{ab-1}$.

% The same process should be done for the rest of the problems.

\end{document}
