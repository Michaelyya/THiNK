\documentclass[10pt]{article}
\usepackage[utf8]{inputenc}
\usepackage[T1]{fontenc}
\usepackage{amsmath}
\usepackage{amsfonts}
\usepackage{amssymb}
\usepackage[version=4]{mhchem}
\usepackage{stmaryrd}

\begin{document}
\section*{Math 141 Tutorial 10 Solutions}
\section*{Main problems}
\begin{enumerate}
  \item Determine whether the following sequences $\left\{a_{n}\right\}_{n=1}^{\infty}$ converge/diverge and justify your answer.\\
(a) $a_{n}=\frac{\ln n}{n}$\\
(d) $a_{n}=2^{-n} \cos (n \pi)$\\
(b) $a_{n}=\frac{n^{2}}{\sqrt{n^{3}+4 n}}$\\
(e) $a_{n}=\left(1+\frac{1}{n}\right)^{n}$\\
(c) $a_{n}=\sin (n \pi)$\\
(f) $a_{n}=\sqrt{n}-\sqrt{n+1} \sqrt{n+3}$
\end{enumerate}

\section*{Solutions:}
(a) Instead of using $a_{n}$, we replace $a_{n}$ with $f(x)=\frac{\ln (x)}{x}$ :

$$
\begin{aligned}
\lim _{n \rightarrow \infty} \frac{\ln n}{n} & =\lim _{n \rightarrow \infty} \frac{\ln x}{x} \\
& =\lim _{n \rightarrow \infty} \frac{\frac{1}{x}}{1}, \quad \text { (L'Hôpital's Rule) } \\
& =\lim _{n \rightarrow \infty} \frac{1}{x} \\
& =0
\end{aligned}
$$

so the sequence converges.\\
(b) We'll compute the limit by factoring out power of $n$ in the numerator and denominator:

$$
\begin{aligned}
\lim _{n \rightarrow \infty} \frac{n^{2}}{\sqrt{n^{3}+4 n}} & =\lim _{n \rightarrow \infty} \frac{n^{2}}{n^{\frac{3}{2}} \sqrt{1+\frac{4}{n^{2}}}} \\
& =\lim _{n \rightarrow \infty} \frac{\sqrt{n}}{\sqrt{1+\frac{4}{n^{2}}}} \\
& =\infty
\end{aligned}
$$

so the sequence diverges.\\
(c) Notice that for every $n \geq 1$, we actually have $\sin (n \pi)=1$ - so the sequence $\left\{a_{n}\right\}_{n=1}^{\infty}$ is constantly zero. Hence

$$
\lim _{n \rightarrow \infty} \sin (n \pi)=\lim _{n \rightarrow \infty} 0=0
$$

so the sequence converges.\\
(d) Notice that we have

$$
\cos (n \pi)=\left\{\begin{array}{ll}
1 & \text { if } n \text { even, } \\
-1 & \text { if } n \text { odd, }
\end{array}=(-1)^{n} .\right.
$$

So, from what we've seen in class we get

$$
\lim _{n \rightarrow \infty} 2^{-n} \cos (n \pi)=\lim _{n \rightarrow \infty} 2^{-n}(-1)^{N}=\lim _{n \rightarrow \infty}\left(\frac{-1}{2}\right)^{n}=0
$$

so the sequence converges.\\
(e) Let

$$
C=\lim _{n \rightarrow \infty} \ln \left(1+\frac{1}{n}\right)^{n}
$$

then we see that

$$
\lim _{n \rightarrow \infty}\left(1+\frac{1}{n}\right)^{n}=e^{C}
$$

It remains to solve for $C$ :

$$
\begin{aligned}
C & =\lim _{n \rightarrow \infty} \ln \left(1+\frac{1}{n}\right)^{n} \\
& =\lim _{n \rightarrow \infty} n \ln \left(1+\frac{1}{n}\right) \\
& =\lim _{n \rightarrow \infty} \frac{\ln \left(1+\frac{1}{n}\right)}{\frac{1}{n}}, \\
& =\lim _{n \rightarrow \infty} \frac{\frac{1}{1+\frac{1}{n}} \frac{-1}{n^{2}}}{\frac{-1}{n^{2}}}, \quad \text { (L'Hôpital's Rule) } \\
& =\lim _{n \rightarrow \infty} \frac{1}{1+\frac{1}{n}} \\
& =1
\end{aligned}
$$

So it follows that

$$
\lim _{n \rightarrow \infty}\left(1+\frac{1}{n}\right)^{n}=e
$$

so the sequence converges.\\
(f) We compute the limit by rationalizing the expression,

$$
\begin{aligned}
\lim _{n \rightarrow \infty} \sqrt{n}-\sqrt{n+1} \sqrt{n+3} & =\lim _{n \rightarrow \infty}(\sqrt{n}-\sqrt{n+1} \sqrt{n+3}) \frac{\sqrt{n}+\sqrt{n+1} \sqrt{n+3}}{\sqrt{n}+\sqrt{n+1} \sqrt{n+3}} \\
& =\lim _{n \rightarrow \infty} \frac{n-\left(n^{2}+4 n+1\right)}{\sqrt{n}+\sqrt{n+1} \sqrt{n+3}}, \\
& =\lim _{n \rightarrow \infty} \frac{-n^{2}-3 n-1}{\sqrt{n}+\sqrt{n+1} \sqrt{n+3}} \\
& =-\lim _{n \rightarrow \infty} \frac{n^{2}\left(1+\frac{3}{n}+\frac{1}{n^{2}}\right)}{n\left(\frac{1}{\sqrt{n}}+\sqrt{1+\frac{1}{n}} \sqrt{1+\frac{3}{n}}\right)} \\
& =-\lim _{n \rightarrow \infty} \frac{n\left(1+\frac{3}{n}+\frac{1}{n^{2}}\right)}{\frac{1}{\sqrt{n}}+\sqrt{1+\frac{1}{n}} \sqrt{1+\frac{3}{n}}} \\
& =-\infty
\end{aligned}
$$

So the sequence diverges.

Page 3\\
2. Compute the limit of the following convergent, recursively defined sequences\\
(a) $a_{n+1}=\frac{1}{2}\left(a_{n}+6\right), a_{0}=2$\\
(b) $a_{n+1}=\frac{a_{n}}{1+a_{n}}, a_{0}=1$\\
(c) $a_{n+1}=\sqrt{2 a_{n}-1}, a_{0}=2$

\section*{Solutions:}
(a) Set

$$
L=\lim _{n \rightarrow \infty} a_{n}
$$

then using the recursive expression for $a_{n+1}$, we get

$$
L=\lim _{n \rightarrow \infty} \frac{1}{2}\left(a_{n}+6\right)=\frac{1}{2}(L+6),
$$

and solving for $L$ we find that $L=6$.\\
(b) Set

$$
L=\lim _{n \rightarrow \infty} a_{n}
$$

then using the recursive expression for $a_{n+1}$, we get

$$
L=\lim _{n \rightarrow \infty} \frac{a_{n}}{1+a_{n}}=\frac{L}{1+L}
$$

and solving for $L$, we get

$$
\begin{aligned}
L=\frac{L}{1+L} & \Longleftrightarrow L^{2}+L=L \\
& \Longleftrightarrow L^{2}=0 \\
& \Longleftrightarrow L=0
\end{aligned}
$$

So the limit of the sequence is 0 .\\
(c) Set

$$
L=\lim _{n \rightarrow \infty} a_{n}
$$

then using the recursive expression for $a_{n+1}$, we get

$$
L=\lim _{n \rightarrow \infty} \sqrt{2 a_{n}-1}=\sqrt{2 L-1}
$$

and solving for $L$, we get

$$
\begin{aligned}
L=\sqrt{2 L-1} & \Longleftrightarrow L^{2}=2 L-1, \\
& \Longleftrightarrow L^{2}-2 L+1=0, \\
& \Longleftrightarrow(L-1)^{2}=0, \\
& \Longleftrightarrow L=1 .
\end{aligned}
$$

So the limit of the sequence is 1 .\\
3. Determine whether the following sequence converges:

$$
a_{n}=\frac{1}{n^{3}} \cos (2 n)
$$

\section*{Solution:}
Using the fact that $-1 \leq \cos (2 n) \leq 1$, we get that

$$
\lim _{n \rightarrow \infty} \frac{-1}{n^{3}} \leq \lim _{n \rightarrow \infty} \frac{\cos (2 n)}{n^{3}} \leq \lim _{n \rightarrow \infty} \frac{1}{n^{3}}
$$

But the limits on both the left and right sides converge:

$$
\lim _{n \rightarrow \infty} \frac{-1}{n^{3}}=0=\lim _{n \rightarrow \infty} \frac{1}{n^{3}}
$$

so by the squeeze theorem it follows that

$$
\lim _{n \rightarrow \infty} \frac{\cos (2 n)}{n^{3}}=0
$$

\begin{enumerate}
  \setcounter{enumi}{3}
  \item Compute the value of the following series:\\
(a) $\sum_{n=1}^{\infty} \frac{1}{n^{2}+3 n+2}$\\
(c) $\sum_{n=1}^{\infty} \frac{1}{\sqrt{n+1}+\sqrt{n}}$\\
(e) $\sum_{n=1}^{\infty} \ln \left(\frac{n+1}{n}\right)$\\
(b) $\sum_{n=1}^{\infty} 2^{n} e^{-n}$\\
(d) $\sum_{n=1}^{\infty} \frac{3^{n+1}}{2^{2 n+2}}$\\
(f) $\sum_{n=1}^{\infty} \frac{1+3^{n}}{4^{n}}$
\end{enumerate}

\section*{Solution:}
(a) We can factor the terms in the sum as

$$
\frac{1}{n^{2}+3 n+2}=\frac{1}{(n+1)(n+2)}
$$

By doing a partial fraction decomposition

$$
\frac{1}{(n+1)(n+2)}=\frac{A}{n+1}+\frac{B}{n+2}
$$

we see that $A(n+2)+B(n+1)=1$, and so we get the system of equations

$$
\begin{aligned}
A+B & =0 \\
2 A+B & =1
\end{aligned}
$$

from which we find that $A=1$ and $B=-1$, i.e.,

$$
\frac{1}{(n+1)(n+2)}=\frac{1}{n+1}-\frac{1}{n+2}
$$

Looking at the sequence of partial sums $S_{k}$, we find that

$$
S_{k}=\sum_{n=1}^{k} \frac{1}{n+1}-\frac{1}{n+2}=\frac{1}{2}-\frac{1}{k+2}
$$

Finally, letting $k \rightarrow \infty$ we compute

$$
\sum_{n=1}^{\infty} \frac{1}{n^{2}+3 n+2}=\lim _{k \rightarrow \infty} S_{k}=\lim _{k \rightarrow \infty} \frac{1}{2}-\frac{1}{k+2}=\frac{1}{2}
$$

(b) Since $2<e$, the sum is a convergent geometric series, and so

$$
\sum_{n=1}^{\infty} 2^{n} e^{-n}=\sum_{n=1}^{\infty}\left(\frac{2}{e}\right)^{n}=\frac{2}{e} \cdot \frac{1}{1-\frac{2}{e}}=\frac{2}{e} \frac{e}{2-e}=\frac{2}{2-e}
$$

(c) We rationalize the terms in the series to find

$$
\frac{1}{\sqrt{n+1}+\sqrt{n}}=\frac{\sqrt{n+1}-\sqrt{n}}{n+1-n}=\sqrt{n+1}-\sqrt{n}
$$

Looking at the sequence of partial sums $S_{k}$, we find that

$$
S_{k}=\sum_{n=1}^{k} \sqrt{n+1}-\sqrt{n}=\sqrt{k+1}-1
$$

and by taking the limit we see that

$$
\sum_{n=1}^{\infty} \frac{1}{\sqrt{n+1}+\sqrt{n}}=\lim _{k \rightarrow \infty} S_{k}=\lim _{k \rightarrow \infty} \sqrt{k+1}-1=\infty
$$

(d) We rewrite the sum as

$$
\sum_{n=1}^{\infty} \frac{3^{n+1}}{2^{2 n+2}}=\frac{3}{4} \sum_{n=1}^{\infty}\left(\frac{3}{4}\right)^{n}
$$

which is a convergent geometric series. Hence we find that

$$
\sum_{n=1}^{\infty} \frac{3^{n+1}}{2^{2 n+2}}=\frac{3}{4} \cdot \frac{3}{4} \frac{1}{1-\frac{3}{4}}=\frac{9}{4}
$$

(e) Using log properties, we can write

$$
\sum_{n=1}^{\infty} \ln \left(\frac{n+1}{n}\right)=\sum_{n=1}^{\infty} \ln (n+1)-\ln (n)
$$

Looking at the sequence of partial sums $S_{k}$, we find that

$$
S_{k}=\sum_{n=1}^{k} \ln (n+1)-\ln (n)=\ln (k+1)
$$

and so

$$
\sum_{n=1} \ln \left(\frac{n+1}{n}\right)=\lim _{k \rightarrow \infty} S_{k}=\lim _{k \rightarrow \infty} \ln (k+1)=\infty
$$

(f) Using properties of sums, we can rewrite the series as

$$
\sum_{n=1}^{\infty} \frac{1+3^{n}}{4^{n}}=\sum_{n=1}^{\infty} \frac{1}{4^{n}}+\sum_{n=1}^{\infty} \frac{3^{n}}{4^{n}}
$$

Both of these are convergent geometric series, so we can compute their values individually:

$$
\begin{aligned}
& \sum_{n=1}^{\infty} \frac{1}{4^{n}}=\frac{1}{4} \frac{1}{1-\frac{1}{4}}=\frac{1}{3} \\
& \sum_{n=1}^{\infty} \frac{3^{n}}{4^{n}}=\frac{3}{4} \frac{1}{1-\frac{3}{4}}=3
\end{aligned}
$$

Putting them together, we get

$$
\sum_{n=1}^{\infty} \frac{1+3^{n}}{4^{n}}=\sum_{n=1}^{\infty} \frac{1}{4^{n}}+\sum_{n=1}^{\infty} \frac{3^{n}}{4^{n}}=\frac{1}{3}+3=\frac{10}{3}
$$


\end{document}