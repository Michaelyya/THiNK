\documentclass[10pt]{article}
\usepackage[utf8]{inputenc}
\usepackage[T1]{fontenc}
\usepackage{amsmath}
\usepackage{amsfonts}
\usepackage{amssymb}
\usepackage[version=4]{mhchem}
\usepackage{stmaryrd}
\usepackage{bbold}

\begin{document}
\section*{Math 141 Tutorial 4 Solutions}
\section*{Main problems}
\begin{enumerate}
  \item In this exercise, you will practice using the Fundamental Theorem of Calculus (Form 2). For every integral below, do the following:
\end{enumerate}

\begin{itemize}
  \item find an antiderivative (i.e. primitive) of the integrand,
  \item evaluate the given integral by applying the FTC if possible. If the FTC does not apply, explain why.\\
(a) $\int_{0}^{1}\left(3 x^{2}+\sqrt{x}-2\right) \mathrm{d} x$\\
(c) $\int_{0}^{\pi} \sec ^{2}(x) \mathrm{d} x$\\
(e) $\int_{-1}^{2}\left((x+1)^{2}+\frac{1}{x}\right) \mathrm{d} x$\\
(b) $\int_{0}^{3 \pi / 2}(\sin (x)+\cos (x)) d x$\\
(d) $\int_{0}^{\frac{1}{\pi}} \frac{1}{1+x^{2}} \mathrm{~d} x$\\
(f) $\int_{1}^{4}\left(3^{x}+1\right) \mathrm{d} x$.
\end{itemize}

Solutions:\\
(a) We have

$$
\int_{0}^{1}\left(3 x^{2}+\sqrt{x}-2\right) \mathrm{d} x=\int_{0}^{1}\left(3 x^{2}+x^{1 / 2}-2\right) \mathrm{d} x .
$$

Notice that

$$
F(x)=x^{3}+\frac{x^{3 / 2}}{3 / 2}-2 x=x^{2}+\frac{2 x^{3 / 2}}{3}-2 x
$$

is an antiderivative of $3 x^{2}+\sqrt{x}-2$. Therefore, since $3 x^{2}+\sqrt{x}-2$ is continuous on $[0,1]$, the Fundamental Theorem tells us that

$$
\int_{0}^{1}\left(3 x^{2}+\sqrt{x}-2\right) \mathrm{d} x=F(1)-F(0)=\left(1+\frac{2}{3}-2\right)-0=-\frac{1}{3} .
$$

(b) Recall that $(\sin x)^{\prime}=\cos (x)$ and $(\cos x)^{\prime}=-\sin x$. Thus, $-\cos x$ is an antiderivative of $\sin x$ and $\sin x$ is an antiderivative of $\cos x$. In particular,

$$
(-\cos x+\sin x)^{\prime}=(-\cos x)^{\prime}+(\sin x)^{\prime}=\sin x+\cos x
$$

Thus, $(-\cos x+\sin x)$ is an antiderivative of the everywhere continuous function $(\sin x+\cos x)$. It follows from the FTC that

$$
\begin{aligned}
\int_{0}^{3 \pi / 2}(\sin (x)+\cos (x)) \mathrm{d} x & =-\cos x+\left.\sin x\right|_{x=0} ^{x=\frac{3 \pi}{2}} \\
& =\left(-\cos \frac{3 \pi}{2}+\sin \frac{3 \pi}{2}\right)-(-\cos 0+\sin 0) \\
& =(-1)-(-1) \\
& =0
\end{aligned}
$$

(c) Recall that $\sec ^{2}(x)$ is the derivative of $\tan (x)$. However, the fundamental theorem does not apply here because $\sec ^{2}(x)$ has an infinite discontinuity at $x=\pi / 2$.\\
(d) Recall that $(\arctan (x))^{\prime}=\frac{1}{1+x^{2}}$, with the latter being continuous on all of $\mathbb{R}$. Therefore, the Fundamental Theorem of Calculus applies:

$$
\int_{0}^{1 / \pi} \frac{1}{1+x^{2}} \mathrm{~d} x=\left.\arctan x\right|_{x=0} ^{x=\frac{1}{\pi}}=\arctan \left(\frac{1}{\pi}\right)-\arctan (0)=\arctan \left(\frac{1}{\pi}\right)
$$

(e) An antiderivative for $(x+1)^{2}+\frac{1}{x}$ is

$$
\frac{1}{3}(x+1)^{3}+\ln |x|, \quad(x \neq 0)
$$

Notice that we can also expand the integrand to $x^{2}+2 x+1+\frac{1}{x}$ before finding an antiderivative.\\
The integrand is discontinuous at $x=0$ (it is not defined at 0 and has a vertical asymptote there), and therefore the fundamental theorem of calculus does not apply on the interval $[-1,2]$.\\
(f) The function

$$
F(x)=\frac{3^{x}}{\ln (3)}+x
$$

is an antiderivative of $3^{x}+1$. Thus, by the FTC,

$$
\begin{aligned}
\int_{1}^{4}\left(3^{x}+1\right) \mathrm{d} x & =\frac{3^{x}}{\ln (3)}+\left.x\right|_{x=1} ^{x=4} \\
& =\frac{3^{4}}{\ln 3}+4-\frac{3^{1}}{\ln (3)}-1 \\
& =\frac{3^{4}-3}{\ln 3}+3 \\
& =\frac{78+3 \ln (3)}{\ln (3)}
\end{aligned}
$$

\begin{enumerate}
  \setcounter{enumi}{1}
  \item Consider a particle moving along a line such that, at any time $t$, the instantaneous velocity of this particle is given by
\end{enumerate}

$$
v(t)=t^{2}-2 t-3, \quad(\mathrm{~m} / \mathrm{s})
$$

(a) Express the displacement of the particle from times $t=2$ to $t=4$ using an integral and evaluate this integral.\\
(b) Express the distance traveled by the particle from times $t=2$ to $t=4$ using an integral and evaluate this integral.\\
Briefly explain the difference between the integrals obtained in (a)-(b). How does this relate to the total area under the curve of a sign-changing function?\\
Solution:\\
(a) Integrating the velocity of the particle from time $t=2$ to $t=4$ gives the total change in position (i.e. the displacement) over this period of time:

$$
\begin{aligned}
\int_{2}^{4}\left(t^{2}-2 t-3\right) \mathrm{d} t & =\frac{t^{3}}{3}-t^{2}-\left.3 t\right|_{t=2} ^{t=4} \\
& =\left(\frac{4^{3}}{3}-4^{2}-3(4)\right)-\left(\frac{2^{3}}{3}-2^{2}-3(2)\right) \\
& =\left(\frac{64}{3}-16-12\right)-\left(\frac{8}{3}-4-6\right) \\
& =\frac{56}{3}-18 \\
& =\frac{56-54}{3} \\
& =\frac{2}{3} \quad \text { (metres). }
\end{aligned}
$$

(b) To obtain the total distance traveled by the particle from time $t=2$ to time $t=4$, we instead want to integrate the speed function, which is the absolute value of the velocity function. Thus, the total distance travelled is

$$
\int_{2}^{4}\left|t^{2}-2 t-3\right| \mathrm{d} t
$$

However, the absolute value function makes this tricky to compute. To circumvent this difficulty, we shall break the integral into two portions: one where the integrand is positive and another where the integrand is negative. After graphing the function (or, alternatively, by noting that $t^{2}-2 t-3=(t-3)(t+1)$, one can find the zeroes of the given function), we see that $t^{2}-2 t-3 \leq 0$ for $0 \leq t \leq 3$, and that $t^{2}-2 t-3 \geq 0$ for $t \geq 3$. Thus, we split our integral as follows:

$$
\begin{aligned}
\int_{2}^{4}\left|t^{2}-2 t-3\right| \mathrm{d} t & =\int_{2}^{3}\left|t^{2}-2 t-3\right| \mathrm{d} t+\int_{3}^{4}\left|t^{2}-2 t-3\right| \mathrm{d} t \\
& =\int_{2}^{3}-\left(t^{2}-2 t-3\right) \mathrm{d} t+\int_{3}^{4}\left(t^{2}-2 t-3\right) \mathrm{d} t
\end{aligned}
$$

Now, each of these integrals can be evaluated using the Fundamental Theorem of Calculus. For instance, the first integral yields

$$
\begin{aligned}
\int_{2}^{3}-\left(t^{2}-2 t-3\right) \mathrm{d} t & =\int_{2}^{3}\left(3+2 t-t^{2}\right) \mathrm{d} t \\
& =3 t+t^{2}-\left.\frac{t^{3}}{3}\right|_{t=2} ^{t=3} \\
& =3(3)+3^{2}-\frac{3^{3}}{3}-\left(3(2)+2^{2}-\frac{2^{3}}{3}\right) \\
& =9-\left(10-\frac{8}{3}\right) \\
& =9-\frac{22}{3} \\
& =\frac{27-22}{3} \\
& =\frac{5}{3}
\end{aligned}
$$

In a similar way, we use the FTC to compute the second integral:

$$
\int_{3}^{4}\left(t^{2}-2 t-3\right) \mathrm{d} t=\frac{7}{3}
$$

Therefore, the total distance traveled by the particle is

$$
\int_{2}^{4}\left|t^{2}-2 t-3\right| \mathrm{d} t=\frac{5}{3}+\frac{7}{3}=\frac{12}{3}=4 \quad \text { (metres). }
$$

\begin{enumerate}
  \setcounter{enumi}{2}
  \item Using the Fundamental Theorem of Calculus (FTC), evaluate the following:\\
(a) $\frac{\mathrm{d}}{\mathrm{d} x} \int_{1}^{x+4}\left(\frac{1}{t}+2\right) \mathrm{d} t$\\
(c) $\frac{\mathrm{d}}{\mathrm{d} x} \int_{\sin (x)}^{2 \pi} \cos (t) d t$\\
(b) $\int_{1}^{x+4}\left(\frac{\mathrm{~d}}{\mathrm{~d} t}\left[\frac{1}{t}+2\right]\right) \mathrm{d} t$\\
(d) $\frac{\mathrm{d}}{\mathrm{d} x} \int_{\sin (x)}^{x^{2}+1} e^{\sqrt[3]{t}} \mathrm{~d} t$.
\end{enumerate}

Solution:\\
(a) Define

$$
F(x)=\int_{1}^{x}\left(\frac{1}{t}+2\right) \mathrm{d} t
$$

and note that $F^{\prime}(x)=\frac{1}{x}+2$ by the fundamental theorem. Therefore, by the Chain Rule,

$$
\begin{aligned}
\frac{\mathrm{d}}{\mathrm{~d} x} \int_{1}^{x+4}\left(\frac{1}{t}+2\right) d t=\frac{\mathrm{d}}{\mathrm{~d} x} F(x+4) & =F^{\prime}(x+4) \frac{\mathrm{d}}{\mathrm{~d} x}(x+4) \\
& =F^{\prime}(x+4) \\
& =\frac{1}{x+4}+2
\end{aligned}
$$

(b) Here we use the second part of the Fundamental Theorem:

$$
\begin{aligned}
\int_{1}^{x+4}\left(\frac{\mathrm{~d}}{\mathrm{~d} t}\left[\frac{1}{t}+2\right]\right) \mathrm{d} t & =\frac{1}{t}+\left.2\right|_{t=1} ^{t=x+4} \\
& =\frac{1}{x+4}-1
\end{aligned}
$$

(c) As in (a), we define a function $F(x)$ as

$$
F(x)=\int_{x}^{2 \pi} \cos (t) \mathrm{d} t=-\int_{2 \pi}^{x} \cos (t) \mathrm{d} t
$$

By the Fundamental Theorem, $F$ is differentiable with derivative

$$
F^{\prime}(x)=-\cos (x)
$$

But then, the Chain Rule implies that

$$
\frac{\mathrm{d}}{\mathrm{~d} x} \int_{\sin (x)}^{2 \pi} \cos (t) d t=\frac{\mathrm{d}}{\mathrm{~d} x} F(\sin (x))=F^{\prime}(\sin (x)) \cos x=-\cos (\sin (x)) \cos x
$$

(d) Consider the function

$$
F(x)=\int_{0}^{x} e^{\sqrt[3]{t}} \mathrm{~d} t
$$

By the Fundamental Theorem, $F^{\prime}(x)=e^{\sqrt[3]{x}}$. Now, observe that

$$
\begin{aligned}
\int_{\sin (x)}^{x^{2}+1} e^{\sqrt[3]{t}} \mathrm{~d} t & =\int_{0}^{x^{2}+1} e^{\sqrt[3]{t}} \mathrm{~d} t-\int_{0}^{\sin x} e^{\sqrt[3]{t}} \mathrm{~d} t \\
& =F\left(x^{2}+1\right)-F(\sin x)
\end{aligned}
$$

Using once more the chain rule, this implies that

$$
\begin{aligned}
\frac{\mathrm{d}}{\mathrm{~d} x} \int_{\sin (x)}^{x^{2}+1} e^{\sqrt[3]{t}} \mathrm{~d} t & =\frac{\mathrm{d}}{\mathrm{~d} x}\left(F\left(x^{2}+1\right)-F(\sin x)\right) \\
& =F^{\prime}\left(x^{2}+1\right) \frac{\mathrm{d}}{\mathrm{~d} x}\left(x^{2}+1\right)-F^{\prime}(\sin x) \frac{\mathrm{d}}{\mathrm{~d} x}(\sin x) \\
& =2 x e^{\sqrt[3]{x^{2}+1}}-\cos x e^{\sqrt[3]{\sin x}}
\end{aligned}
$$

\begin{enumerate}
  \setcounter{enumi}{3}
  \item Compute the following integrals using substitution:\\
(a) $\int_{0}^{2} 2 e^{3 s+5} \mathrm{~d} s$\\
(e) $\int_{0}^{1} \sqrt{2 t+1} d t$\\
(b) $\int 2 x \cos ((x-1)(x+1)) \mathrm{d} x$\\
(c) $\int_{0}^{\frac{\pi}{2}} \frac{\cos (x)}{1+\sin (x)} \mathrm{d} x$\\
(f) $\int e^{\cos (x)} \sin (x) d x$\\
(d) $\int \frac{2 x^{3}+1}{\left(x^{4}+2 x\right)^{3}} \mathrm{~d} x$\\
(g) $\int_{0}^{1} t^{4}\left(1+t^{5}\right)^{10} d t$
\end{enumerate}

Solution:\\
(a) We define $u:=3 s+5$ so that $\mathrm{d} u=3 \mathrm{~d} s$. Now, $u(0)=3(0)+5=5$ and $u(3)=3(2)+5=11$. Our integral therefore becomes

$$
\begin{aligned}
\int_{0}^{2} 2 e^{3 s+5} \mathrm{~d} s=\int_{5}^{11} 2 e^{u} \frac{\mathrm{~d} u}{3} & =\frac{2}{3} \int_{5}^{11} e^{u} \mathrm{~d} u \\
& =\left.\frac{2}{3} e^{u}\right|_{u=5} ^{u=11} \\
& =\frac{2\left(e^{11}-e^{5}\right)}{3}
\end{aligned}
$$

(b) Notice that this integral is precisely

$$
\int 2 x \cos ((x-1)(x+1)) \mathrm{d} x=\int 2 x \cos \left(x^{2}-1\right) \mathrm{d} x .
$$

Take $u:=x^{2}-1$ so that $\mathrm{d} u=2 x \mathrm{~d} x$. Thus,

$$
\begin{aligned}
\int 2 x \cos ((x-1)(x+1)) \mathrm{d} x & =\int 2 x \cos \left(x^{2}-1\right) \mathrm{d} x \\
& =\int \cos (u) \mathrm{d} u \\
& =\sin (u)+C \\
& =\sin \left(x^{2}-1\right)+C
\end{aligned}
$$

(c) Put $u:=1+\sin (x)$ and note that $\mathrm{d} u=\cos (x) \mathrm{d} x$. Moreover, $u(0)=1$ and $u(\pi / 2)=2$. Therefore, our given integral becomes

$$
\begin{aligned}
\int_{0}^{\frac{\pi}{2}} \frac{\cos (x)}{1+\sin (x)} \mathrm{d} x & =\int_{1}^{2} \frac{\mathrm{~d} u}{u} \\
& =\ln (2)-\ln (1) \\
& =\ln 2
\end{aligned}
$$

(d) Define $u:=x^{4}+2 x$ so that $\mathrm{d} u=\left(4 x^{3}+2\right) \mathrm{d} x=2\left(2 x^{3}+1\right) \mathrm{d} x$. Thus, our given integral becmomes

$$
\begin{aligned}
\int \frac{2 x^{3}+1}{\left(x^{4}+2 x\right)^{3}} \mathrm{~d} x & =\frac{1}{2} \int \frac{\mathrm{~d} u}{u^{3}} \\
& =-\frac{1}{4 u^{2}}+C \\
& =-\frac{1}{4\left(x^{4}+2 x\right)^{2}}+C
\end{aligned}
$$

(e) Put $u:=2 t+1$, then $\mathrm{d} u=2 \mathrm{~d} t$. Moreover, $u(0)=1$ and $u(1)=3$. Therefore, our given integral becomes

$$
\begin{aligned}
\int_{0}^{1} \sqrt{2 t+1} \mathrm{~d} t & =\int_{1}^{3} \frac{\sqrt{u}}{2} \mathrm{~d} u \\
& =\left.\frac{1}{2} \cdot \frac{2}{3} u^{3 / 2}\right|_{1} ^{3} \\
& =\sqrt{3}-\frac{1}{3}
\end{aligned}
$$

(f) Set $u:=\cos (x)$, then $\mathrm{d} u=-\sin (x) \mathrm{d} x$. Hence,

$$
\begin{aligned}
\int e^{\cos (x)} \sin (x) \mathrm{d} x & =-\int e^{u} \mathrm{~d} u \\
& =-e^{u}+C \\
& =-e^{\cos (x)}+C
\end{aligned}
$$

(g) Put $u:=1+t^{5}$, then $\mathrm{d} u=5 t^{4} \mathrm{~d} t$ and rearranging this gives us $t^{4} \mathrm{~d} t=\mathrm{d} u / 5$. Moreover, $u(0)=1$ and $u(1)=2$. Thus, our given integral becomes

$$
\begin{aligned}
\int_{0}^{1} t^{4}\left(1+t^{5}\right)^{10} \mathrm{~d} t & =\frac{1}{5} \int_{1}^{2} u^{1} 0 \mathrm{~d} u \\
& =\left.\frac{1}{5} \cdot \frac{u^{11}}{11}\right|_{1} ^{2} \\
& =\frac{2^{11}-1}{55}
\end{aligned}
$$

\begin{enumerate}
  \setcounter{enumi}{4}
  \item Suppose that $f$ is a continuous function on $\mathbb{R}$. Prove that $\int_{a}^{b} f(x+c) \mathrm{d} x=\int_{a+c}^{b+c} f(x) \mathrm{d} x$ for any $c \in \mathbb{R}$.\\
Solution:
\end{enumerate}

Fix $c \in \mathbb{R}$ and consider the function $g(x):=x+c$ on $\mathbb{R}$. This function is differentiable for all $x$ with continuous derivative $g^{\prime}(x)=1$. Hence, our given integral can be written as

$$
\int_{a}^{b} f(x+c) \mathrm{d} x=\int_{a}^{b} f(g(x)) \mathrm{d} x=\int_{a}^{b} f(g(x)) \underbrace{g^{\prime}(x)}_{=1} \mathrm{~d} x .
$$

By substitution, this is the same as

$$
\begin{aligned}
\int_{a}^{b} f(x+c) \mathrm{d} x & =\int_{a}^{b} f(g(x)) g^{\prime}(x) \mathrm{d} x \\
& =\int_{g(a)}^{g(b)} f(u) \mathrm{d} u \\
& =\int_{a+c}^{b+c} f(u) \mathrm{d} u .
\end{aligned}
$$

\section*{Challenge Problem}
\begin{enumerate}
  \setcounter{enumi}{5}
  \item Compute
\end{enumerate}

$$
I=\int_{0}^{\pi} \frac{x \sin (x)}{1+\cos ^{2}(x)} d x
$$

$\underline{\text { Hint: }}$ make the substitution $u=\pi-x$ and with trig identities write down a simple definite integral for $2 I$


\end{document}