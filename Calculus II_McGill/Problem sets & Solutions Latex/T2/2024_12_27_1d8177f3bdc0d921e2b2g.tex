\documentclass[10pt]{article}
\usepackage[utf8]{inputenc}
\usepackage[T1]{fontenc}
\usepackage{amsmath}
\usepackage{amsfonts}
\usepackage{amssymb}
\usepackage[version=4]{mhchem}
\usepackage{stmaryrd}
\usepackage{bbold}

\begin{document}
\section*{Math 141 Tutorial 2}
\section*{Main problems}
\begin{enumerate}
  \item Using the summation formulas seen in class, provide a closed form for the following summations in terms of $n$.\\
(a) $\sum_{i=0}^{n}(i+1)$\\
(d) $\sum_{i=0}^{n}(i-2)^{2}$\\
(b) $\sum_{i=2}^{n}(2 i+n)$\\
(e) $\sum_{i=-n}^{0}-i^{3}$\\
(c) $\sum_{i=-1}^{n}(i+2)^{2}$\\
(f) $\sum_{i=-n}^{n}\left(i^{3}+3 i^{2} n+3 i n^{2}+n^{3}\right)$
  \item Evaluate the definite integrals by either taking the limit of (left or right) Riemann sums or interpreting the integral as an area.\\
(a) $\int_{0}^{2} 2 x \mathrm{~d} x$\\
(d) $\int_{1}^{5}\left(x^{2}+2 x\right) \mathrm{d} x$\\
(b) $\int_{1}^{4}(3-x) \mathrm{d} x$\\
(e) $\int_{0}^{3} x^{3} \mathrm{~d} x$\\
(c) $\int_{0}^{2}\left(2 x^{2}+1\right) \mathrm{d} x$\\
(f) $\int_{-2}^{2}|2 x| \mathrm{d} x$
  \item Suppose that $f, g:[a, b] \rightarrow \mathbb{R}$ are continuous functions and let $k \in \mathbb{R}$ be a constant. By using the corresponding properties for summations, prove the following\\
(a) $\int_{a}^{b}(k f(x)) \mathrm{d} x=k \int_{a}^{b} f(x) \mathrm{d} x$\\
(b) $\int_{a}^{b}(f(x)+g(x)) \mathrm{d} x=\int_{a}^{b} f(x) \mathrm{d} x+\int_{a}^{b} g(x) \mathrm{d} x$
\end{enumerate}

\section*{Challenge problems}
\begin{enumerate}
  \setcounter{enumi}{3}
  \item Using Riemann sums, show that
\end{enumerate}

$$
\int_{a(x)}^{b(x)} t \mathrm{~d} t=\frac{(b(x))^{2}-(a(x))^{2}}{2}
$$

\begin{enumerate}
  \setcounter{enumi}{4}
  \item What do you expect the value of
\end{enumerate}

$$
\int_{-\pi / 2}^{3 \pi / 2} \cos x \mathrm{~d} x
$$

to be? Hint: use symmetry.\\
6. Let $f(x)=x^{2}$.\\
(a) Using Riemann sums, determine the function

$$
F(x)=\int_{0}^{x} f(t) \mathrm{d} t
$$

(b) Determine the derivative $F^{\prime}(x)$ of $F(x)$. How does this function relate to $f$ ?


\end{document}