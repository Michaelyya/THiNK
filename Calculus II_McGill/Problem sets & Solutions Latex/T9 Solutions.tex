\documentclass[10pt]{article}
\usepackage[utf8]{inputenc}
\usepackage[T1]{fontenc}
\usepackage{amsmath}
\usepackage{amsfonts}
\usepackage{amssymb}
\usepackage[version=4]{mhchem}
\usepackage{stmaryrd}
\usepackage{bbold}

\begin{document}
\section*{Math 141 Tutorial 9 Solutions}
\section*{Main problems}
\begin{enumerate}
  \item Determine whether each improper integral is convergent or divergent and justify your statement. When convergent, is it always possible to compute the value of the improper integral? If so, compute it.\\
(a) $\int_{e}^{\infty} \frac{1}{x(\ln (x))^{2}} \mathrm{~d} x$\\
(b) $\int_{0}^{\infty} \frac{1}{x^{2}+2 x+2} \mathrm{~d} x$\\
(c) $\int_{-1}^{1} \frac{e^{1 / x}}{x^{2}} \mathrm{~d} x$
\end{enumerate}

\section*{Solution:}
(a) By definition,

$$
\int_{e}^{\infty} \frac{1}{x(\ln (x))^{2}} \mathrm{~d} x=\lim _{t \rightarrow \infty} \int_{e}^{t} \frac{1}{x(\ln (x))^{2}} \mathrm{~d} x
$$

With the substitution $u=\ln (x)$, we compute

$$
\int_{e}^{t} \frac{1}{x(\ln (x))^{2}} \mathrm{~d} x=\int_{1}^{\ln (t)} \frac{1}{u^{2}} \mathrm{~d} x=-\left.\frac{1}{u}\right|_{1} ^{\ln (t)}=-\frac{1}{\ln (t)}+1
$$

It follows that

$$
\int_{e}^{\infty} \frac{1}{x(\ln (x))^{2}} \mathrm{~d} x=\lim _{t \rightarrow \infty}\left(-\frac{1}{\ln (t)}+1\right)=1
$$

In particular, the integral converges.\\
(b) By definition,

$$
\int_{0}^{\infty} \frac{1}{x^{2}+2 x+2} \mathrm{~d} x=\lim _{t \rightarrow \infty} \int_{0}^{t} \frac{1}{x^{2}+2 x+2} \mathrm{~d} x
$$

With the substitution $u=x+1$, we compute

$$
\begin{aligned}
\int_{0}^{t} \frac{1}{x^{2}+2 x+2} \mathrm{~d} x=\int_{0}^{t} \frac{1}{(x+1)^{2}+1} \mathrm{~d} x & =\int_{1}^{t+1} \frac{1}{u^{2}+1} \mathrm{~d} u \\
& =\left.\arctan (u)\right|_{1} ^{t+1} \\
& =\arctan (t+1)-\arctan (1) \\
& =\arctan (t+1)-\frac{\pi}{4}
\end{aligned}
$$

It follows that

$$
\int_{0}^{\infty} \frac{1}{x^{2}+2 x+2} \mathrm{~d} x=\lim _{t \rightarrow \infty}\left(\arctan (t+1)-\frac{\pi}{4}\right)=\frac{\pi}{2}-\frac{\pi}{4}=\frac{\pi}{4}
$$

In particular, the integral converges.\\
(c) We begin by finding an antiderivative of $\frac{e^{1 / x}}{x^{2}}$. With the substitution $u=1 / x$, we find

$$
\int \frac{e^{1 / x}}{x^{2}} \mathrm{~d} x=-\int e^{u} \mathrm{~d} u=-e^{u}+C=-e^{1 / x}+C
$$

In particular, an antiderivative of $\frac{e^{1 / x}}{x^{2}}$ is $-e^{1 / x}$.\\
This computation aside, we solve the main problem. Notice that

$$
\frac{e^{1 / x}}{x^{2}}
$$

is not defined at $x=0$. Hence, we have an improper integral that we must split into two parts:

$$
\int_{-1}^{1} \frac{e^{1 / x}}{x^{2}} \mathrm{~d} x=\int_{-1}^{0} \frac{e^{1 / x}}{x^{2}} \mathrm{~d} x+\int_{0}^{1} \frac{e^{1 / x}}{x^{2}} \mathrm{~d} x
$$

By definition

$$
\int_{-1}^{0} \frac{e^{1 / x}}{x^{2}} \mathrm{~d} x=\lim _{t \rightarrow 0^{-}} \int_{-1}^{t} \frac{e^{1 / x}}{x^{2}} \mathrm{~d} x
$$

Since, by our initial remarks, we know that an antiderivative of $\frac{e^{1 / x}}{x^{2}}$ is $-e^{1 / x}$,

$$
\int_{-1}^{t} \frac{e^{1 / x}}{x^{2}} \mathrm{~d} x=-e^{1 / t}-\left(-e^{1 /(-1)}\right)=-e^{1 / t}+\frac{1}{e}
$$

Thus,

$$
\int_{-1}^{0} \frac{e^{1 / x}}{x^{2}} \mathrm{~d} x=\lim _{t \rightarrow 0^{-}}\left(-e^{1 / t}+\frac{1}{e}\right)=\frac{1}{e}
$$

Here, we have used that $\lim _{t \rightarrow 0^{-}} 1 / t=-\infty$ and $\lim _{x \rightarrow-\infty} e^{x}=0$ to conclude that $\lim _{t \rightarrow 0^{-}} e^{1 / t}=0$.\\
Similarly, we evaluate

$$
\begin{aligned}
\int_{0}^{1} \frac{e^{1 / x}}{x^{2}} \mathrm{~d} x & =\lim _{t \rightarrow 0^{+}} \int_{t}^{1} \frac{e^{1 / x}}{x^{2}} \mathrm{~d} x \\
& =\lim _{t \rightarrow 0^{+}}\left(-e^{1 / 1}-\left(-e^{1 / t}\right)\right) \quad=\lim _{t \rightarrow 0^{+}}\left(-e^{1 / 1}+e^{1 / t}\right)=+\infty
\end{aligned}
$$

In the last equality, we have used that $\lim _{t \rightarrow 0^{+}} 1 / t=+\infty$ and $\lim _{x \rightarrow+\infty} e^{x}=+\infty$ to conclude that $\lim _{t \rightarrow 0^{+}} e^{1 / t}=+\infty$.\\
In conclusion, the integral

$$
\int_{-1}^{1} \frac{e^{1 / x}}{x^{2}} \mathrm{~d} x=\int_{-1}^{0} \frac{e^{1 / x}}{x^{2}} \mathrm{~d} x+\int_{0}^{1} \frac{e^{1 / x}}{x^{2}} \mathrm{~d} x
$$

diverges since the second part diverges.\\
2. Use the comparison test to determine whether each of the following integrals converge or diverge.\\
(a) $\int_{1}^{\infty} \frac{1+e^{-\cos x}}{\sqrt{x}} \mathrm{~d} x$\\
(b) $\int_{1}^{\infty} \frac{x}{x^{3}+1} \mathrm{~d} x$\\
(c) $\int_{0}^{\infty} \frac{\arctan (x)}{2+e^{x}} \mathrm{~d} x$

\section*{Solution:}
(a) Since $e^{z}>0$ for all $z \in \mathbb{R}$, we have the inequality

$$
\frac{1+e^{-\cos (x)}}{\sqrt{x}} \geq \frac{1}{\sqrt{x}}=\frac{1}{x^{1 / 2}} \geq 0
$$

for all $x \in \mathbb{R}$. Especially, this inequality holds for all $x \geq 1$. Now, we know from class ( $p$-test) that the integral

$$
\int_{1}^{\infty} \frac{1}{\sqrt{x}} \mathrm{~d} x=\int_{1}^{\infty} \frac{1}{x^{1 / 2}} \mathrm{~d} x
$$

is divergent because $1 / 2 \leq 1$. Hence, by the comparison theorem, the integral

$$
\int_{1}^{\infty} \frac{1+e^{-\cos (x)}}{\sqrt{x}}
$$

is divergent.\\
(b) Here, notice that $x^{3}+1 \geq x^{3}$ for all $x \in \mathbb{R}$. In particular, we have $x^{3}+1 \geq x^{3}$ for all $x \geq 1$. Hence, there holds

$$
0 \leq \frac{1}{x^{3}+1} \leq \frac{1}{x^{3}} \quad \text { for all } x \geq 1
$$

This implies that

$$
0 \leq \frac{x}{x^{3}+1} \leq \frac{x}{x^{3}}=\frac{1}{x^{2}} \quad \text { for all } x \geq 1
$$

Because

$$
\int_{1}^{\infty} \frac{1}{x^{2}} \mathrm{~d} x
$$

is convergent, it follows from the comparison theorem that

$$
\int_{1}^{\infty} \frac{x}{x^{3}+1} \mathrm{~d} x
$$

also converges.\\
(c) Recall that $0 \leq \arctan (x) \leq \frac{\pi}{2}$ for all $x \geq 0$. Therefore,

$$
0 \leq \frac{\arctan (x)}{2+e^{x}} \mathrm{~d} x \leq \frac{\pi / 2}{2+e^{x}} \leq \frac{\pi}{2 e^{x}}=\frac{\pi}{2} e^{-x}
$$

In this last step, we have used that $2+e^{x}>e^{x}$ for all $x \in \mathbb{R}$. If we can show that

$$
\int_{0}^{\infty} \frac{\pi}{2} e^{-x} \mathrm{~d} x=\frac{\pi}{2} \int_{0}^{\infty} e^{-x} \mathrm{~d} x
$$

converges, it will then follow from the comparison test that

$$
\int_{0}^{\infty} \frac{\arctan (x)}{2+e^{x}} \mathrm{~d} x
$$

converges. Now,

$$
\begin{aligned}
\int_{0}^{\infty} e^{-x} \mathrm{~d} x & =\lim _{t \rightarrow \infty} \int_{0}^{t} e^{-x} \mathrm{~d} x \\
& =\lim _{t \rightarrow \infty}\left(-\left.e^{-x}\right|_{x=0} ^{x=t}\right) \\
& =\lim _{t \rightarrow \infty}\left(1-e^{-t}\right) \\
& =1
\end{aligned}
$$

By our earlier remarks, we infer that the improper integral

$$
\int_{0}^{\infty} \frac{\arctan (x)}{2+e^{x}} \mathrm{~d} x
$$

is convergent.\\
3. Determine whether each improper integral is convergent or divergent and justify your statement. Note that you do not need to evaluate each integral.\\
(a) $\int_{\pi}^{\infty} \frac{\cos (\ln (x))+2}{x^{1 / 4}} \mathrm{~d} x$\\
(b) $\int_{e}^{\infty} \frac{\ln (x)}{x} \mathrm{~d} x$\\
(c) $\int_{5}^{\infty} \frac{1}{\sqrt{x-\sqrt{x}}} \mathrm{~d} x$

\section*{Solution:}
(a) We shall use the comparison test to show that this improper integral is divergent. First, using that $\cos (z) \geq-1$ for all $z \in \mathbb{R}$, we have

$$
\cos (\ln (x))+2 \geq-1+2=1, \quad \text { for all } x>0
$$

In particular, one has

$$
\frac{\cos (\ln (x))+2}{x^{1 / 4}} \geq \frac{1}{x^{1 / 4}} \geq 0
$$

for all $x \in[\pi, \infty)$. Now, notice that

$$
\underbrace{\int_{1}^{\infty} \frac{1}{x^{1 / 4}} \mathrm{~d} x}_{\text {divergent }}=\underbrace{\int_{1}^{\pi} \frac{1}{x^{1 / 4}} \mathrm{~d} x}_{<\infty}+\int_{\pi}^{\infty} \frac{1}{x^{1 / 4}} \mathrm{~d} x
$$

This implies that the improper integral

$$
\int_{\pi}^{\infty} \frac{1}{x^{1 / 4}} \mathrm{~d} x
$$

is also divergent. One can also check directly that this integral diverges by verifying that

$$
\lim _{t \rightarrow \infty} \int_{\pi}^{t} \frac{1}{x^{1 / 4}} \mathrm{~d} x=\infty
$$

In any case, because

$$
\frac{\cos (\ln (x))+2}{x^{1 / 4}} \geq \frac{1}{x^{1 / 4}} \geq 0
$$

for all $x \in[\pi, \infty)$, the comparison theorem implies that the improper integral

$$
\int_{\pi}^{\infty} \frac{\cos (\ln (x))+2}{x^{1 / 4}} \mathrm{~d} x
$$

is divergent.\\
(b) Let us attempt to directly evaluate this improper integral. By definition,

$$
\int_{e}^{\infty} \frac{\ln (x)}{x} \mathrm{~d} x=\lim _{t \rightarrow \infty} \int_{e}^{t} \frac{\ln (x)}{x} \mathrm{~d} x
$$

Next, let us handle the integral

$$
\int_{e}^{t} \frac{\ln (x)}{x} \mathrm{~d} x
$$

By making the substitution $u:=\ln (x)$ with $\mathrm{d} u=\frac{1}{x} \mathrm{~d} x$, this becomes

$$
\begin{aligned}
\lim _{t \rightarrow \infty} \int_{e}^{t} \frac{\ln (x)}{x} \mathrm{~d} x & =\lim _{t \rightarrow \infty} \int_{1}^{\ln (t)} u \mathrm{~d} u \\
& =\lim _{t \rightarrow \infty}\left(\left.\frac{u^{2}}{2}\right|_{u=1} ^{u=\ln (t)}\right) \\
& =\lim _{t \rightarrow \infty} \frac{1}{2}\left(\ln (t)^{2}-1\right) \\
& =\infty
\end{aligned}
$$

where in this last step we have used that

$$
\lim _{t \rightarrow \infty} \ln (t)^{2}=\infty
$$

Hence,

$$
\int_{e}^{\infty} \frac{\ln (x)}{x} \mathrm{~d} x=\lim _{t \rightarrow \infty} \int_{e}^{t} \frac{\ln (x)}{x} \mathrm{~d} x
$$

is divergent.\\
(c) Our last integral to consider is $\int_{5}^{\infty} \frac{1}{\sqrt{x-\sqrt{x}}} \mathrm{~d} x$. Observe that, for $x \geq 5$

$$
\sqrt{x} \geq \sqrt{5}
$$

Hence

$$
\sqrt{x-\sqrt{x}} \leq \sqrt{x-\sqrt{5}}
$$

Taking the reciprocal on either sider, we obtain the inequality

$$
\frac{1}{\sqrt{x-\sqrt{x}}} \geq \frac{1}{\sqrt{x-\sqrt{5}}} \geq 0
$$

By the comparison test, if

$$
\int_{5}^{\infty} \frac{1}{\sqrt{x-\sqrt{5}}} \mathrm{~d} x
$$

diverges, then so does

$$
\int_{5}^{\infty} \frac{1}{\sqrt{x-\sqrt{x}}} \mathrm{~d} x
$$

By definition,

$$
\begin{aligned}
\int_{5}^{\infty} \frac{1}{\sqrt{x-\sqrt{5}}} \mathrm{~d} x & =\lim _{t \rightarrow \infty} \int_{5}^{t} \frac{1}{\sqrt{x-\sqrt{5}}} \mathrm{~d} x \\
& =\lim _{t \rightarrow \infty} \int_{5-\sqrt{5}}^{t-\sqrt{5}} \frac{1}{\sqrt{u}} \mathrm{~d} u \\
& =\lim _{t \rightarrow \infty}\left[\frac{2}{3} u^{3 / 2}\right]_{5-\sqrt{5}}^{t-\sqrt{5}} \\
& =\frac{2}{3} \lim _{t \rightarrow \infty}\left((t-\sqrt{5})^{3 / 2}-(5-\sqrt{5})^{3 / 2}\right) \\
& =+\infty
\end{aligned}
$$

By our earlier remarks, we conclude that

$$
\int_{5}^{\infty} \frac{1}{\sqrt{x-\sqrt{x}}} \mathrm{~d} x
$$

diverges.

Page 7\\
4. Compute the arc length of the following curves over the given interval.\\
(a) $y=\frac{2}{3} x^{3 / 2}$ for $x \in[0,2]$\\
(b) $y=\ln (\sec (x))$ for $x \in[0, \pi / 4]$\\
(c) $y=\frac{x^{3}}{3}+\frac{1}{4 x}$ for $x \in[1,2]$\\
(d) $x=\sqrt{y-y^{2}}+\arcsin (\sqrt{y})$ for $y \geq \frac{1}{2}$.

\section*{Solution:}
(a) We have that $y^{\prime}=x^{1 / 2}$, so

$$
L=\int_{0}^{2} \sqrt{1+y^{\prime}} \mathrm{d} x=\int_{0}^{2} \sqrt{1+x} \mathrm{~d} x .
$$

Using the substitution $u=1+x$, we get

$$
L=\int_{1}^{3} \sqrt{u} \mathrm{~d} u=\left.\frac{2}{3} u^{\frac{3}{2}}\right|_{1} ^{3}=\frac{2}{3}\left(3^{3 / 2}-1\right) .
$$

(b) We have that

$$
y^{\prime}=\frac{1}{\sec (x)} \sec (x) \tan (x)=\tan (x)
$$

so

$$
\begin{aligned}
L & =\int_{0}^{\frac{\pi}{4}} \sqrt{1+\tan ^{2}(x)} \\
& =\int_{0}^{\frac{\pi}{4}} \sec (x) \mathrm{d} x \\
& =\left.\ln |\sec (x)+\tan (x)|\right|_{0} ^{\frac{\pi}{4}} \\
& =\ln |\sqrt{2}+1|-\ln |1| \\
& =\ln |\sqrt{2}+1|
\end{aligned}
$$

(c) We have that

$$
y^{\prime}=x^{2}-\frac{1}{4 x^{2}}
$$

SO

$$
\begin{aligned}
L & =\int_{1}^{2} \sqrt{1+\left(x^{2}-\frac{1}{4 x^{2}}\right)^{2}} \mathrm{~d} x \\
& =\int_{1}^{2} \sqrt{1+x^{4}+\frac{1}{16 x^{2}}-\frac{1}{2}} \mathrm{~d} x \\
& =\int_{1}^{2} \sqrt{x^{4}+\frac{1}{16 x^{2}}+\frac{1}{2}} \mathrm{~d} x \\
& =\int_{1}^{2} \sqrt{\left(x^{2}+\frac{1}{4 x^{2}}\right)^{2}} \mathrm{~d} x \\
& =\int_{1}^{2} x^{2}+\frac{1}{4 x^{2}} \mathrm{~d} x \\
& =\frac{x^{3}}{3}-\left.\frac{1}{4 x}\right|_{1} ^{2} \\
& =59 / 24
\end{aligned}
$$

(d) We have that

$$
y^{\prime}=\frac{1}{2 \sqrt{y-y^{2}}}(1-2 y)+\frac{1}{\sqrt{1-(\sqrt{y})^{2}}} \frac{1}{2 \sqrt{y}}=\frac{1-y}{\sqrt{y(1-y)}}=\sqrt{\frac{1-y}{y}}
$$

so

$$
\begin{aligned}
L & =\int_{\frac{1}{2}}^{\infty} \sqrt{1+\left(\sqrt{\frac{1-y}{y}}\right)^{2}} \mathrm{~d} y \\
& =\int_{\frac{1}{2}}^{\infty} \sqrt{1+\frac{1-y}{y}} \mathrm{~d} y \\
& =\int_{\frac{1}{2}}^{\infty} \frac{1}{\sqrt{y}} \mathrm{~d} y \\
& =\lim _{b \rightarrow \infty} \int_{\frac{1}{2}}^{b} \frac{1}{\sqrt{y}} \mathrm{~d} y \\
& =\left.\lim _{b \rightarrow \infty} 2 \sqrt{y}\right|_{\frac{1}{2}} ^{b} \\
& =\infty
\end{aligned}
$$

\begin{enumerate}
  \setcounter{enumi}{4}
  \item Find a function $f(x)$ whose arc length between $x=0$ and $x=1 / 2$ is given by
\end{enumerate}

$$
\int_{0}^{\frac{1}{2}} \frac{1+x^{2}}{1-x^{2}} \mathrm{~d} x
$$

Then, compute this arc length integral.

\section*{Solution:}
Answer: $f(x)=\ln \left(1-x^{2}\right)$ and

$$
\int_{0}^{\frac{1}{2}} \frac{1+x^{2}}{1-x^{2}} \mathrm{~d} x=\ln (3)-\frac{1}{2}
$$

\begin{enumerate}
  \setcounter{enumi}{5}
  \item Evaluate the area of the surface obtained by rotating the given curve about the specified axis.\\
(a) $y=\sqrt{1+e^{x}}$ for $x \in[0,1]$ about the $x$-axis.\\
(b) $y=\frac{1}{4} x^{2}-\frac{1}{2} \ln (x)$ for $x \in[1,2]$ about the $x$-axis.\\
(c) $x=1-y^{2}$ for $y \in[0,3]$ about the $x$-axis.\\
(d) $x=1+\sqrt{1-y^{2}}, 0 \leq y \leq 1 / 2$ about the $y$-axis.
\end{enumerate}

Solution:\\
(a) Since we are rotating around the $x$-axis, we will integrate with respect to $x$. Also, we have

$$
y^{\prime}=\frac{1}{2 \sqrt{1+e^{x}}} e^{x}
$$

so

$$
\begin{aligned}
S A & =2 \pi \int_{0}^{1} y \sqrt{1+\left(y^{\prime}\right)^{2}} \mathrm{~d} x \\
& =2 \pi \int_{0}^{1} \sqrt{1+e^{x}} \sqrt{1+\left(\frac{1}{2 \sqrt{1+e^{x}}} e^{x}\right)^{2}} \mathrm{~d} x \\
& =2 \pi \int_{0}^{1} \sqrt{\left(1+e^{x}\right)\left(1+\frac{1}{4\left(1+e^{x}\right)} e^{2 x}\right)} \mathrm{d} x \\
& =2 \pi \int_{0}^{1} \sqrt{1+e^{x}+\frac{1}{4} e^{2 x}} \mathrm{~d} x \\
& =\frac{\pi}{2} \int_{0}^{1} \sqrt{\left(2+e^{x}\right)^{2}} \mathrm{~d} x \\
& =\frac{\pi}{2} \int_{0}^{1} 2+e^{x} \mathrm{~d} x \\
& =\left.\frac{\pi}{2}\left(2 x+e^{x}\right)\right|_{0} ^{1} \\
& =\frac{\pi}{2}(1+e)
\end{aligned}
$$

(b) We have

$$
y^{\prime}=\frac{1}{2} x-\frac{1}{2 x}
$$

and since we're rotating around the $x$-axis, we have the same setup as in part $(a)$,

$$
\begin{aligned}
S A & =2 \pi \int_{1}^{2}\left(\frac{1}{4} x^{2}-\frac{1}{2} \ln (x)\right) \sqrt{1+\left(\frac{1}{2} x-\frac{1}{2 x}\right)^{2}} \mathrm{~d} x \\
& =2 \pi \int_{1}^{2}\left(\frac{1}{4} x^{2}-\frac{1}{2} \ln (x)\right) \sqrt{1+\frac{x^{2}}{4}-\frac{1}{2}+\frac{1}{4 x^{2}}} \mathrm{~d} x \\
& =2 \pi \int_{1}^{2}\left(\frac{1}{4} x^{2}-\frac{1}{2} \ln (x)\right) \sqrt{\frac{x^{2}}{4}+\frac{1}{2}+\frac{1}{4 x^{2}}} \mathrm{~d} x \\
& =2 \pi \int_{1}^{2}\left(\frac{1}{4} x^{2}-\frac{1}{2} \ln (x)\right) \sqrt{\left(\frac{1}{2} x-\frac{1}{2 x}\right)^{2}} \mathrm{~d} x \\
& =2 \pi \int_{1}^{2}\left(\frac{1}{4} x^{2}-\frac{1}{2} \ln (x)\right)\left(\frac{1}{2} x-\frac{1}{2 x}\right) \mathrm{d} x \\
& =2 \pi \int_{1}^{2} \frac{x^{3}}{8}+\frac{x}{8}+\frac{\ln (x)}{4 x}-\frac{1}{4} x \ln (x) \mathrm{d} x \\
& =\left.2 \pi\left(\frac{x^{4}}{32}-\frac{1}{8} x^{2} \ln (x)+\frac{(\ln (x))^{2}}{8}\right)\right|_{1} ^{2} \\
& =\frac{1}{32}\left((\ln (4))^{2}-15\right) .
\end{aligned}
$$

(c) Since we're integrating around the $x$-axis, again we integrate with respect to $x$. Rearranging the given function to write it in terms of $x$, we have

$$
y= \pm \sqrt{1-x}^{2}
$$

and in our case, we pick $y=\sqrt{1-x^{2}}$ since the question states that $y \in[0,3]$ (so in particular it's non-negative). Moreover, we can figure out the bounds of the integral by plugging in the values $y_{1}=0$ and $y_{2}=3$ into the given functions:

$$
\begin{aligned}
& x_{1}=1-y_{1}^{2}=1 \\
& x_{2}=1-y_{2}^{2}=1-3^{2}=-8
\end{aligned}
$$

Finally, differentiating the expression $y=\sqrt{1-x}$ gives

$$
y^{\prime}=\frac{-1}{2 \sqrt{1-x}}
$$

Putting it all together, we get

$$
\begin{aligned}
S A & =2 \pi \int_{-8}^{1} y \sqrt{1+\left(y^{\prime}\right)^{2}} \mathrm{~d} x \\
& =2 \pi \int_{-8}^{1} \sqrt{1-x} \sqrt{1+\frac{1}{4(1-x)}} \mathrm{d} x \\
& =2 \pi \int_{-8}^{1} \sqrt{1-x+\frac{1}{4}} \mathrm{~d} x \\
& =2 \pi \int_{-8}^{1} \sqrt{\frac{5}{4}-x} \mathrm{~d} x
\end{aligned}
$$

Using the substitution $v^{2}=\frac{5}{4}-x$, it follows that $\mathrm{d} x=-2 v \mathrm{~d} v$ and

$$
\begin{aligned}
S A & =-2 \pi \int_{\frac{\sqrt{37}}{2}}^{\frac{1}{2}} \sqrt{v^{2}} 2 v \mathrm{~d} v \\
& =4 \pi \int_{\frac{1}{2}}^{\frac{\sqrt{37}}{2}} v^{2} \mathrm{~d} v \\
& =\left.4 \pi \frac{1}{3} v^{3}\right|_{\frac{1}{2}} ^{\frac{\sqrt{37}}{2}} \\
& =\frac{\pi}{9}\left(37^{\frac{3}{2}}-1\right)
\end{aligned}
$$

(d) Since we're rotating about the $y$-axis, we integrate with respect to $y$. We have

$$
x^{\prime}=\frac{1}{2 \sqrt{1-y^{2}}} \cdot(-2 y)=\frac{-y}{\sqrt{1-y^{2}}}
$$

so

$$
\begin{aligned}
S A & =2 \pi \int_{0}^{\frac{1}{2}} x \sqrt{1+\left(x^{\prime}\right)^{2}} \mathrm{~d} y \\
& =2 \pi \int_{0}^{\frac{1}{2}}\left(1+\sqrt{1-y^{2}}\right) \sqrt{1+\frac{y^{2}}{1-y^{2}}} \mathrm{~d} y \\
& =2 \pi \int_{0}^{\frac{1}{2}}\left(1+\sqrt{1-y^{2}}\right) \sqrt{\frac{1}{1-y^{2}}} \mathrm{~d} y \\
& =2 \pi \int_{0}^{\frac{1}{2}} \frac{1}{\sqrt{1-y^{2}}}+1 \mathrm{~d} y \\
& =\left.2 \pi(\arcsin y+y)\right|_{0} ^{\frac{1}{2}} \\
& =2 \pi\left(\frac{\pi}{6}+\frac{1}{2}\right)
\end{aligned}
$$

\section*{Extra Practice Problems}
\begin{enumerate}
  \item Determine why each of the following integrals is improper. Can you determine whether each integral converges or diverges?\\
(a) $\int_{25}^{\infty} \frac{1}{x^{6}+x+45} d x$\\
(d) $\int_{-\infty}^{\infty} x e^{-x^{2}} d x$\\
(g) $\int_{1}^{\infty} \frac{e^{1 / x}}{x^{3}} d x$\\
(b) $\int_{0}^{1} \ln (x) \mathrm{d} x$\\
(e) $\int_{5}^{\infty} \frac{3+e^{-2 x}}{5 x} d x$\\
(h) $\int_{0}^{\pi / 2} \frac{\cos \theta}{\sqrt{\sin \theta}} \mathrm{~d} \theta$\\
(c) $\int_{0}^{1} \frac{1}{\sqrt{1-x^{2}}} \mathrm{~d} x$\\
(f) $\int_{-\infty}^{-1} \frac{e^{1 / x}}{x^{3}} d x$\\
(i) $\int_{-\infty}^{0} \frac{z+1}{z-2} \mathrm{~d} z$
\end{enumerate}

\section*{Challenge Problems}
\begin{enumerate}
  \item Show that $\int_{0}^{\infty} \frac{\ln (x)}{1+x^{2}} d x=0$\\
$\underline{\text { Hint: }}$ : you might find the substitution $x=e^{t}$ useful.
  \item Suppose $f^{\prime}(x)=\frac{1}{x^{2}+(f(x))^{2}}$ and that $f(1)=1$.\\
(a) Show that $f(x)$ is an increasing function on $[1, \infty)$.\\
(b) Deduce from part (a) that $\int_{1}^{\infty} f^{\prime}(x) d x \leq \int_{1}^{\infty} \frac{1}{x^{2}+1} d x$.\\
(c) Using (a) and (b), prove that $\lim _{x \rightarrow \infty} f(x) \leq 1+\frac{\pi}{4}$.
  \item Determine whether $\int_{-1}^{1} e^{1 / x} \mathrm{~d} x$ converges or diverges. Can you justify your answer using class results?
\end{enumerate}

\end{document}