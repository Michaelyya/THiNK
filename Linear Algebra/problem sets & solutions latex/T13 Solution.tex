\documentclass[10pt]{article}
\usepackage[utf8]{inputenc}
\usepackage[T1]{fontenc}
\usepackage{amsmath}
\usepackage{amsfonts}
\usepackage{amssymb}
\usepackage[version=4]{mhchem}
\usepackage{stmaryrd}
\usepackage{bbold}

\begin{document}
\section*{Lecture hours 30-32}
\section*{Definitions and Theorems}
Definition (Eigenvectors - Eigenvalues ). A vector $\vec{v} \in \mathbb{R}^{n}$ is an eigenvector of $n \times n$ matrix $A$ if there exists a scalar $\lambda$ ("lambda") such that $A \vec{v}=\lambda \vec{v}$, and $\vec{v} \neq \overrightarrow{0}$. This $\lambda$ is called the corresponding eigenvalue.

Definition (Eigenspace). If $A$ is an $n \times n$ matrix and $\lambda$ is a scalar, the $\lambda$-eigenspace of $A$ (denoted $E_{\lambda}$ ) is the set of all vector $\vec{v} \in \mathbb{R}^{n}$ such that $A \vec{v}=\lambda \vec{v}$. So, the nonzero vector in $E_{\lambda}$ are exactly the eigenvector of $A$ with eigenvalues $\lambda$. This set is a subspace.

Definition (Geometric Multiplicity). The geometric multiplicity of an eigenvalue $\lambda$ is the dimension of its eigenspace $(\operatorname{dim}(\operatorname{ker}(\lambda \mathrm{I}-A))$ ).

Definition (Algebraic Multiplicity). The algebraic multiplicity of an eigenvalue $\lambda_{1}$ is the number of times the factor $\left(\lambda-\lambda_{1}\right)$ appears in the characteristic polynomial $c_{A}(\lambda) \stackrel{\text { def }}{=} \operatorname{det}(\lambda I-A)$.

Definition (Diagonalization). We say a $n \times n$ matrix $A$ is diagonalizable if there exists an invertible matrix $S$ and a diagonal matrix $B$ such that $A=S B S^{-1}$.

Problem 52. For what values of $\lambda$ is the following matrix invertible?

$$
\left[\begin{array}{cccc}
\lambda & 0 & 0 & 0 \\
0 & 1-\lambda & -2 & 7 \\
0 & -1 & 2-\lambda & 3 \\
0 & 0 & 0 & 4-\lambda
\end{array}\right]
$$

Solution 52 The determinant of this matrix may be computed to be

$$
-\lambda^{2}(\lambda-3)(\lambda-4)
$$

A matrix is invertible if and only if its determinant is nonzero. Therefore, this matrix is invertible if and only if $\lambda$ is not equal to 0,3 , or 4 .

Problem 53. For which values of $k$ does the matrix

$$
\left[\begin{array}{cc}
k+1 & k \\
-k & 1-k
\end{array}\right]
$$

have an eigenbasis?

Solution 53 The characteristic polynomial of this matrix is $(\lambda-1)^{2}$, so for all values of $k$ it has an eigenvalue $\lambda=1$ with algebraic multiplicity 2 . On the other hand, the kernel of the matrix

$$
I-A=\left[\begin{array}{cc}
-k & -k \\
k & k
\end{array}\right]
$$

is 2-dimensional only when $k=0$. Therefore, this matrix has an eigenbasis only when $k=0$.

Problem 54 (Eigenvalues). Two $n \times n$ matrices $A$ and $B$ are called similar if there is an invertible matrix $S$ with $A=S B S^{-1}$.\\
a) Show that if $X$ and $Y$ are similar, then $\operatorname{det} X=\operatorname{det} Y$.\\
b) Show that if $A$ is similar to $B$, then the matrix $\lambda I-A$ is similar to $\lambda I-B$.\\
c) Use the previous two parts to show that similar matrices have the same eigenvalues. (Hint: use the characteristic polynomial).

\section*{Solution 54 (Eigenvalues and similar matrices)}
a) If $X$ and $Y$ are similar, then $X=S Y S^{-1}$. Then

$$
\begin{aligned}
\operatorname{det} X & =\operatorname{det}\left(S Y S^{-1}\right) \\
& =\operatorname{det} S \operatorname{det} Y \operatorname{det}\left(S^{-1}\right) \\
& =\frac{\operatorname{det} S}{\operatorname{det} S} \operatorname{det} Y \\
& =\operatorname{det} Y
\end{aligned}
$$

b) If $A$ is similar to $B$, then $A=S B S^{-1}$. Now,

$$
\begin{aligned}
S(\lambda I-A) S^{-1} & =S \lambda I S^{-1}-S A S^{-1} \\
& =\lambda S S^{-1}-B \\
& =\lambda I-B,
\end{aligned}
$$

so $\lambda I-A$ is similar to $\lambda I-B$.\\
c) By combining parts a and b , if $A$ and $B$ are similar, then

$$
\operatorname{det}(\lambda I-A)=\operatorname{det}(\lambda I-B)
$$

Therefore the characteristic polynomials of $A$ and $B$ are the same, so $A$ and $B$ must have the same eigenvalues (with the same algebraic multiplicities).

Problem 55 (Diagonalization and Dynamical Systems). The matrix $A$ has the following eigenvectors and eigenvalues: $\vec{v}_{1}=\left[\begin{array}{l}1 \\ 0 \\ 2\end{array}\right]$ with eigenvalue $\lambda_{1}=1, \vec{v}_{2}=$ $\left[\begin{array}{c}1 \\ -1 \\ 0\end{array}\right]$ with eigenvalue $\lambda_{2}=-1$, and $\vec{v}_{3}=\left[\begin{array}{l}0 \\ 1 \\ 1\end{array}\right]$ with eigenvalue $\lambda_{3}=2$.\\
a) Diagonalise $A$, i.e., find a matrix $S$ such that $S^{-1} A S$ is a diagonal matrix.\\
b) Find a closed form for $A^{2021}$.\\
c) Consider the discrete dynamical system given by $\vec{x}(t+1)=A \vec{x}(t)$ with initial condition $\vec{x}(0)=\left[\begin{array}{l}1000 \\ 2000 \\ 3000\end{array}\right]$. Find $x(2021)$.

Solution 55 (Diagonalization and Dynamical Systems)\\
a) Let $S$ be the matrix whose columns are the eigenvectors of $A$ :

$$
\left[\begin{array}{ccc}
1 & 1 & 0 \\
0 & -1 & 1 \\
2 & 0 & 1
\end{array}\right] .
$$

By expressing $A$ in terms of the eigenbasis given by $\left\{\vec{v}_{1}, \vec{v}_{2}, \vec{v}_{3}\right\}$, we have that $A=S D S^{-1}$, where $D$ is the diagonal $3 \times 3$ matrix with entries $1,-1,2$.\\
b) From the previous part, we know that $A=S D S^{-1}$. It follows that $A^{2021}=$ $S D^{2021} S^{-1}$, so

$$
\begin{aligned}
A^{2021} & =\left[\begin{array}{ccc}
1 & 1 & 0 \\
0 & -1 & 1 \\
2 & 0 & 1
\end{array}\right]\left[\begin{array}{ccc}
1 & 0 & 0 \\
0 & (-1)^{2021} & 0 \\
0 & 0 & 2^{2021}
\end{array}\right]\left[\begin{array}{ccc}
1 & 1 & 0 \\
0 & -1 & 1 \\
2 & 0 & 1
\end{array}\right]^{-1} \\
& =\left[\begin{array}{ccc}
1 & 1 & 0 \\
0 & -1 & 1 \\
2 & 0 & 1
\end{array}\right]\left[\begin{array}{ccc}
1 & 0 & 0 \\
0 & 1 & 0 \\
0 & 0 & 2^{2021}
\end{array}\right]\left[\begin{array}{ccc}
-1 & -1 & 1 \\
2 & 1 & -1 \\
2 & 2 & -1
\end{array}\right] \\
& =\left[\begin{array}{ccc}
1 & 0 & 0 \\
2\left(2^{2021}-1\right) & 2^{2021}-1 & 1-2^{2021} \\
2\left(2^{2021}-1\right) & 2\left(2^{2021}-1\right) & 2-2^{2021}
\end{array}\right]
\end{aligned}
$$

c) We know that $\vec{x}(2021)$ is given by $A^{2021} \vec{x}(0)$.


\end{document}