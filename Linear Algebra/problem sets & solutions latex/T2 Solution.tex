\documentclass[10pt]{article}
\usepackage[utf8]{inputenc}
\usepackage[T1]{fontenc}
\usepackage{amsmath}
\usepackage{amsfonts}
\usepackage{amssymb}
\usepackage[version=4]{mhchem}
\usepackage{stmaryrd}
\usepackage{bbold}

\begin{document}
\section*{Lecture hours 3-4}
\section*{Definitions}
Definition (Rank of a matrix). The rank of a matrix is the number of leading ones in the rref of that matrix.

Definition (Linear Combination). A linear combination of the vectors $\bar{v}_{1}, \bar{v}_{2}, \ldots, \bar{v}_{n}$ is an expression of the form $c_{1} \bar{v}_{1}+c_{2} \bar{v}_{2}+\cdots+c_{n} \bar{v}_{n}$ where $c_{1}, c_{2}$, through $c_{n}$ are real numbers. So it's just a sum of multiples of the vectors $\bar{v}_{1}, \bar{v}_{2}, \ldots, \bar{v}_{n}$.

Definition (Span). The span of the vectors $\bar{v}_{1}, \bar{v}_{2}, \ldots, \bar{v}_{n}$ is all possible linear combinations of these vectors, and it is denoted by $\operatorname{span}\left(\bar{v}_{1}, \bar{v}_{2}, \ldots, \bar{v}_{n}\right)$.

Definition (Homogeneous System). A homogeneous system of linear equations is a system in which each equation has no constant term.

Problem 6 (Rank of a coefficient matrix). Suppose you have a system of three linear equations for two unknowns.\\
a) What is the largest possible rank the coefficient matrix could have? What is the smallest possible rank?\\
b) If the system is consistent, what is the largest possible number of free variables in the solution? What is the smallest possible number?\\
c) What are the possibilities for the number of solutions?

Now suppose you have a different system, this time there are three linear equations for four unknowns.\\
d) What is the largest possible rank the coefficient matrix could have? What is the smallest possible rank?\\
e) If the system is consistent, what is the largest possible number of free variables in the solution? What is the smallest possible number?\\
f) What are the possibilities for the number of solutions?

Solution 6 (Rank of a coefficient matrix) Remember that in the augmented matrix of a linear system, rows correspond to equations, and columns correspond to variables in the system. For this problem, try giving examples for each one of the cases.\\
a) The largest possible rank is 2 , the smallest is 0 .\\
b) The largest number is 2 , the smallest is 0 .\\
c) There could be zero, one, or infinitely many solutions.\\
d) The largest possible rank is 3 , the smallest is 0 .\\
e) The largest number is 4 , the smallest is 1 .\\
f) There could be zero or infinitely many solutions.

Problem 7 (Linear systems with parameters). For the linear system

$$
\begin{aligned}
x-y+2 z & =4, \\
3 x-2 y+9 z & =14, \\
2 x-4 y+a z & =b,
\end{aligned}
$$

find real numbers $a$ and $b$ such that:\\
a) The system has a unique solution.\\
b) The system has infinitely many solutions.\\
c) The system is inconsistent.

Solution 7 (Linear systems with parameters) We can solve the system by Gaussian elimination:\\
a) If we want an unique solution: to get a leading 1 for the third row in the rref, we need to be able to multiply by $\frac{1}{a+2}$. Thus we can take any $a \neq-2$.\\
b) For infinitely many solutions: we need the third row in the rref to be a row of zeros. Thus we can take $a=-2$ and $b=4$.\\
c) For an inconsistent system: take $a=-2$ and $b \neq 4$.

Problem 8 (Span 1). Is the vector

$$
\left[\begin{array}{l}
1 \\
0
\end{array}\right]
$$

in the span of the vectors

$$
\vec{u}_{1}=\left[\begin{array}{c}
1 \\
-2
\end{array}\right], \vec{u}_{2}=\left[\begin{array}{l}
3 \\
5
\end{array}\right] ?
$$

Are there vectors in $\mathbb{R}^{2}$ that are not in the span of $\vec{u}_{1}$ and $\vec{u}_{2}$ ? Explain why or why not.

Solution 8 (Span 1) To determine if this vector is in the span of $\vec{u}_{1}$ and $\vec{u}_{2}$ we ask if we can find $x$ and $y$ such that

$$
x \vec{u}_{1}+y \vec{u}_{2}=\left[\begin{array}{l}
1 \\
0
\end{array}\right] .
$$

This becomes the system of linear equations

$$
\begin{array}{r}
x+3 y=1, \\
-2 x+5 y=0,
\end{array}
$$

which can be solved by the usual method to give $x=5 / 11, y=2 / 11$. So the vector is in the span of $\vec{u}_{1}$ and $\vec{u}_{2}$. Every vector is in the span of $\vec{u}_{1}$ and $\vec{u}_{2}$, because these two vectors do not lie on the same line.

Problem 9 (Span 2). Consider the three vectors in $\mathbb{R}^{3}$ :

$$
\vec{v}_{1}=\left[\begin{array}{l}
1 \\
1 \\
0
\end{array}\right], \vec{v}_{2}=\left[\begin{array}{l}
1 \\
t \\
0
\end{array}\right], \vec{v}_{3}=\left[\begin{array}{c}
-1 \\
-1 \\
s
\end{array}\right]
$$

where $s, t \in \mathbb{R}$. What are the values of $s$ and $t$ so that $\vec{v}_{1}, \vec{v}_{2}$, and $\vec{v}_{3}$ span:\\
a) A line.\\
b) A plane.\\
c) All of $\mathbb{R}^{3}$.

\section*{Solution 9 (Span 2 )}
a) To span a line, $\vec{v}_{1}$ and $\vec{v}_{2}$ must lie on the same line, and thus must be multiples of each other, so $t=1$. We also must have $\vec{v}_{1}$ and $\vec{v}_{3}$ multiples of each other, so $s=0$.\\
b) If $s \neq 0$, then $\vec{v}_{3}$ is definitely not in the span of $\vec{v}_{1}$ and $\vec{v}_{2}$, so the only way to have the span of all three be a plane is if the span of $\vec{v}_{1}$ and $\vec{v}_{2}$ is a line, i.e. if $t=1$. If $s=0$, then the span of $\vec{v}_{1}$ and $\vec{v}_{3}$ is a line, so the span of all three will be a plane if $v_{2}$ does not lie on this line, i.e. if $t \neq 1$. So there are two possibilities: $s=0, t \neq 1$, and $s \neq 0, t=1$.\\
c) The span of three non-zero vectors is either a line, plane or all of $\mathbb{R}^{3}$, and we just analysed the other two possibilities. Thus, the condition is $s \neq 0, t \neq 1$.

Problem 10 (Homogeneous systems). Suppose you have a homogeneous system of three equations for three unknowns $x, y$, and $z$. The coefficient matrix of this system has rank 3. What is the solution? Why?

\section*{Solution 10 (Homogeneous systems)}
A consistent system will have a unique solution if the rank equals the number of columns. For this problem, the system has three equations and is rank 3, so we know there is a unique solution. We also know that $x=y=z=0$ is a solution, because the system is homogeneous. Thus, the only solution is $x=y=z=0$

Problem 11. (Linear combinations) The vectors $\vec{x}$ and $\vec{y}$ are in the span of the vectors $\vec{w}_{1}$ and $\vec{w}_{2}$. The vector $\vec{z}$ is a linear combination of $\vec{x}$ and $\vec{y}$. Is $\vec{z}$ in the span of $\vec{w}_{1}$ and $\vec{w}_{2}$ ? Why or why not?

Solution 11 (Linear combinations) It is in the span of $\vec{w}_{1}$ and $\vec{w}_{2}$, because we know:

$$
\begin{aligned}
& \vec{z}=c_{1} \vec{x}+c_{2} \vec{y}, \\
& \vec{x}=c_{3} \vec{w}_{1}+c_{4} \vec{w}_{2}, \\
& \vec{y}=c_{5} \vec{w}_{1}+c_{6} \vec{w}_{2},
\end{aligned}
$$

for some $c_{1}, \ldots, c_{6} \in \mathbb{R}$, so

$$
\vec{z}=\left(c_{1} c_{3}+c_{2} c_{5}\right) \vec{w}_{1}+\left(c_{1} c_{4}+c_{2} c_{6}\right) \vec{w}_{2}
$$


\end{document}