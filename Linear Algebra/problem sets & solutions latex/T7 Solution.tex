\documentclass[10pt]{article}
\usepackage[utf8]{inputenc}
\usepackage[T1]{fontenc}
\usepackage{amsmath}
\usepackage{amsfonts}
\usepackage{amssymb}
\usepackage[version=4]{mhchem}
\usepackage{stmaryrd}
\usepackage{bbold}

\begin{document}
\section*{Lecture hours 13-15}
\section*{Definitions and Theorems}
Definition (Kernel and Image of a linear transformation). Let $T: \mathbb{R}^{n} \rightarrow \mathbb{R}^{m}$ be a linear transformation

\begin{itemize}
  \item The kernel $\operatorname{ker}(\mathrm{T})$ is the set of vectors $\vec{x} \in \mathbb{R}^{n}$ such that $T(\vec{x})=0$.
  \item The image of T is the set of all vectors $\vec{y} \in \mathbb{R}^{m}$ such that $T(\vec{x})=\vec{y}$ for some $\vec{x} \in \mathbb{R}^{n}$.
\end{itemize}

Definition (Rank and Nullity of a linear transformation). Let $T: \mathbb{R}^{n} \rightarrow \mathbb{R}^{m}$ be a linear transformation

\begin{itemize}
  \item The rank of T is the dimension of the image of $\mathrm{T}, \operatorname{rank} \mathrm{T}=\operatorname{dim}(\mathrm{im} \mathrm{T})$.
  \item The nullity of T is the dimension of the kernel of T, nullity $\mathrm{T}=\operatorname{dim}(\operatorname{ker} \mathrm{T})$.
\end{itemize}

\section*{Theorem (Rank Nullity Theorem).}
\begin{itemize}
  \item In terms of linear transformations:
\end{itemize}

Let $T: \mathbb{R}^{n} \rightarrow \mathbb{R}^{m}$ be a linear transformation

$$
\operatorname{rank} \mathrm{T}+\text { nullity } \mathrm{T}=n
$$

\begin{itemize}
  \item In terms of matrices:
\end{itemize}

Let $A$ be an $m \times n$ matrix

$$
\operatorname{dim}(\operatorname{im} A)+\operatorname{dim}(\operatorname{ker} A)=\text { number of columns of } \mathrm{A}=n .
$$

Problem 29 (Rank and Nullity). Let $\vec{v} \neq \overrightarrow{0}$ be the vector $\vec{v}=\left[\begin{array}{c}v_{1} \\ v_{2} \\ v_{3}\end{array}\right]$. Define a linear transformation $T: \mathbb{R}^{3} \rightarrow \mathbb{R}^{3}$ by

$$
T(\vec{x})=\vec{v} \times \vec{x}
$$

a) What is the nullity of T ?\\
b) What is the rank of $T$ ? Why?

Solution 29 (Rank and Nullity)\\
a) From Problem 27 in Tutorial 6 we know that if $\vec{x}$ is in the kernel of T then $x=c \vec{v}$ for some $c \in \mathbb{R}$.

From the definition of cross product, it is straightforward to show that any scalar multiple of vector $\vec{v}$ is in the kernel of T.

Therefore, $\operatorname{Ker} \mathrm{T}=\operatorname{span}(\vec{v})$ and nullity $\mathrm{T}=1$.\\
b) The rank of $T$ is 2 . The easiest way to find this is to use that the kernel of $T$ is 1-dimensional and apply the rank-nullity theorem.

Problem 30 (Rank Nullity Theorem). True or false? Justify your answer.\\
a) If $A$ is a $2 \times 4$ matrix with kernel of dimension 2 , then the equation $A \vec{x}=\vec{e}_{2}$ is consistent.\\
b) There is a $5 \times 5$ matrix $A$ such that $\operatorname{dim}(\mathrm{imA})=\operatorname{dim}(\operatorname{kerA})$.

Solution 30 (Rank Nullity Theorem)\\
a) True, by the rank-nullity theorem $A$ has rank 2 , so the image of $A$ is $\mathbb{R}^{2}$.\\
b) False. Suppose $\operatorname{dim}(\mathrm{im} A)=\operatorname{dim}($ ker A$)$ for some $5 \times 5$ matrix A . By the rank-nullity theorem we have

$$
5=\operatorname{dim}(\operatorname{im} A)+\operatorname{dim}(\operatorname{ker} A)=2 \operatorname{dim}(\operatorname{im} A) .
$$

which is impossible because $2 \operatorname{dim}(\operatorname{imA})$ is always an even integer. It cannot be equal to 5 .

Problem 31 (Rank Nullity Theorem). Let $T: \mathbb{R}^{4} \rightarrow \mathbb{R}^{3}$ be the linear transformation defined by

$$
T\left(\left[\begin{array}{l}
a \\
b \\
c \\
d
\end{array}\right]\right)=\left[\begin{array}{l}
a-b \\
c-d
\end{array}\right]
$$

Find the kernel, nullity, image and rank of T.

Solution 31 (Rank Nullity Theorem)

\begin{enumerate}
  \item If $\left[\begin{array}{l}a \\ b \\ c \\ d\end{array}\right]$ is in the kernel of T then $a=b$ and $c=d$. It follows that the kernel of T is given by the span of $\left\{\left[\begin{array}{l}1 \\ 1 \\ 0 \\ 0\end{array}\right],\left[\begin{array}{l}0 \\ 0 \\ 1 \\ 1\end{array}\right]\right\}$ and nullity $\mathrm{T}=2$.
  \item From the Rank Nullity Theorem we have
\end{enumerate}

$$
\operatorname{rank} \mathrm{T}+\text { nullity } \mathrm{T}=\operatorname{rank} \mathrm{T}+2=4
$$

Therefore rank $\mathrm{T}=2$. In other words, the image of T is a 2-dimensional subspace of $\mathbb{R}^{2}$. It follows that imT $=\mathbb{R}^{2}$.

Problem 32 (Rank Nullity Theorem). Let $T: \mathbb{R}^{3} \rightarrow \mathbb{R}^{2}$ be the linear transformation defined by

$$
T\left(\left[\begin{array}{l}
a \\
b \\
c
\end{array}\right]\right)=\left[\begin{array}{l}
a \\
b
\end{array}\right]
$$

Find the kernel, nullity, image and rank of T.\\
Solution 32 (Rank Nullity Theorem)

\begin{enumerate}
  \item If $\left[\begin{array}{l}a \\ b \\ c\end{array}\right]$ is in the kernel of T then $a=0$ and $c=0$. It follows that the kernel of T is given by the span of $\left[\begin{array}{l}0 \\ 0 \\ 1\end{array}\right]$ and nullity $\mathrm{T}=1$.
  \item From the Rank Nullity Theorem we can conclude that the image of T is a 2-dimensional subspace of $\mathbb{R}^{2}$. It follows that imT $=\mathbb{R}^{2}$.
\end{enumerate}

\end{document}