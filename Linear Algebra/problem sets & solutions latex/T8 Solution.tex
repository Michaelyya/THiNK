\documentclass[10pt]{article}
\usepackage[utf8]{inputenc}
\usepackage[T1]{fontenc}
\usepackage{amsmath}
\usepackage{amsfonts}
\usepackage{amssymb}
\usepackage[version=4]{mhchem}
\usepackage{extpfeil}
\usepackage{stmaryrd}
\usepackage{bbold}

\begin{document}
\section*{Lecture hours 16-18}
\section*{Definitions and Theorems}
Definition (Matrix multiplication). Let A be an $m \times p$ and B an $p \times n$. Matrix AB is an $m \times n$ matrix. Recall, to multiply matrices together: multiply left matrix by each column of the right matrix, those are the columns of the resulting matrix.

Definition (Composition of linear transformations). Let $S: \mathbb{R}^{m} \rightarrow \mathbb{R}^{p}$ and $T: \mathbb{R}^{p} \rightarrow$ $\mathbb{R}^{n}$ be linear transformations. The composition $T \circ S: \mathbb{R}^{m} \rightarrow \mathbb{R}^{n}$ (pronounced "T composed with $\mathrm{S}^{\prime \prime}$ ) is given by

$$
T \circ S(\vec{v}) \xlongequal{\text { def }} T(S(\vec{v})),
$$

for $\vec{v} \in \mathbb{R}^{m}$.

Definition (Invertible linear transformation).

In terms of linear transformations:\\
A linear transformation $T: \mathbb{R}^{m} \rightarrow \mathbb{R}^{n}$ is invertible if for all $\vec{y} \in \mathbb{R}^{n}$ (outputs of T ) there exists an $\vec{x} \in \mathbb{R}^{m}$ such that $T(\vec{x})=\vec{y}$, and this $\vec{x}$ is unique.

In terms of matrices:\\
A matrix $A$ is invertible if for all $\vec{y} \in \mathbb{R}^{n}$ there is an unique $\vec{x} \in \mathbb{R}^{m}$ such that $A \vec{x}=\vec{y}$.

Problem 33 (Matrix algebra). Compute the following matrix products:\\
a)

$$
\left[\begin{array}{l}
1 \\
2 \\
3
\end{array}\right]\left[\begin{array}{ll}
4 & 5
\end{array}\right] .
$$

b)

$$
\left[\begin{array}{lll}
1 & 2 & 3
\end{array}\right]\left[\begin{array}{l}
4 \\
5 \\
6
\end{array}\right]
$$

Solution 33 (Matrix algebra)\\
a) This operation is called outer product.

$$
\left[\begin{array}{l}
1 \\
2 \\
3
\end{array}\right]\left[\begin{array}{ll}
4 & 5
\end{array}\right]=\left[\begin{array}{ll}
1 \cdot 4 & 1 \cdot 5 \\
2 \cdot 4 & 2 \cdot 5 \\
3 \cdot 4 & 3 \cdot 5
\end{array}\right]=\left[\begin{array}{cc}
4 & 5 \\
8 & 10 \\
12 & 15
\end{array}\right]
$$

b) This operation is called inner product.

$$
\left[\begin{array}{lll}
1 & 2 & 3
\end{array}\right]\left[\begin{array}{l}
4 \\
5 \\
6
\end{array}\right]=4 \cdot 1+2 \cdot 5+3 \cdot 6=32
$$

Problem 34 (Compositions and inverses). Let $T: \mathbb{R}^{3} \rightarrow \mathbb{R}^{3}$ be projection from $\mathbb{R}^{3}$ onto the $x y$-plane.\\
a) Is $T$ invertible? Why or why not?\\
b) Find the matrix $A$ with $T(\vec{x})=A \vec{x}$.\\
c) Find a linear transformation $S: \mathbb{R}^{3} \rightarrow \mathbb{R}^{3}$ with $T \circ S=S \circ T$.\\
d) Find a linear transformation $S: \mathbb{R}^{3} \rightarrow \mathbb{R}^{3}$ with $T \circ S \neq S \circ T$.

Solution 34 (Compositions and inverses)\\
a) $T$ is not invertible, because $\operatorname{ker}(T) \neq\{0\}$.\\
b) $A=\left[\begin{array}{lll}1 & 0 & 0 \\ 0 & 1 & 0 \\ 0 & 0 & 0\end{array}\right]$.\\
c) The identity transformation.\\
d) $S=$ rotation of 90 degrees around the $x$-axis works. The vector $\vec{e}_{2}$ is in the kernel of $T \circ S$, but $S \circ T\left(\vec{e}_{2}\right)=\vec{e}_{3}$. There are many other possible answers too.

Problem 35 (Compositions). True or false? If true, explain why; if false, give a counterexample.\\
a) If $T: \mathbb{R}^{3} \rightarrow \mathbb{R}^{3}$ is a linear transformation and $\operatorname{ker}(T)=\{\overrightarrow{0}\}$, then $T$ is invertible.\\
b) If $T: \mathbb{R}^{2} \rightarrow \mathbb{R}^{2}$ is a linear transformation with nullity 1 and $S: \mathbb{R}^{2} \rightarrow$ $\mathbb{R}^{2}$ is another linear transformation with nullity 1 , then $T \circ S$ is the zero transformation.\\
c) If $T$ and $S$ are linear transformations with the domain of $T$ equal to the codomain of $S$, then $\operatorname{rank}(T \circ S) \leq \operatorname{rank}(T)$.\\
d) If $T$ and $S$ are linear transformations with the domain of $T$ equal to the codomain of $S$, then $\operatorname{rank}(T \circ S) \leq \operatorname{rank}(S)$.

Solution 35 (Compositions)\\
a) True. If $\operatorname{ker}(T)=\{0\}$, then, by the rank-nullity theorem, the rank of $T$ is 3 , so the image of $T$ is $\mathbb{R}^{3}$. Therefore, the equation $T(\vec{x})=\vec{y}$ has a solution for any $\vec{y} \in \mathbb{R}^{3}$. This solution is unique because $\operatorname{ker}(T)=\{0\}$.\\
b) This is false. For example, both $T$ and $S$ could be projection to the $x$-axis.\\
c) This is true, because any vector that is in the image of $T \circ S$ must be in the image of $T$.\\
d) This is true. Let $A$ be the matrix for $T$ and $B$ be the matrix for $S$. Let the columns of $B$ be $\vec{b}_{1}, \ldots, \vec{b}_{n}$. We have:

$$
\operatorname{image}(S)=\operatorname{span}\left(\vec{b}_{1}, \ldots, \vec{b}_{n}\right), \quad \operatorname{image}(T \circ S)=\operatorname{span}\left(A \vec{b}_{1}, \ldots, A \vec{b}_{n}\right)
$$

Any linear relation among the $\vec{b}_{1}, \ldots, \vec{b}_{n}$ gives a linear relation among the $A \vec{b}_{1}, \ldots, A \vec{b}_{n}$ (because $A$ represents a linear transformation). Therefore, the dimension of $\operatorname{span}\left(A \vec{b}_{1}, \ldots, A \vec{b}_{n}\right)$ cannot be larger than the dimension of $\operatorname{span}\left(\vec{b}_{1}, \ldots, \vec{b}_{n}\right)$.

Problem 36 (Inverse of a problem). Let $A=\left[\begin{array}{ll}0 & 1 \\ 1 & 0\end{array}\right]$ and $B=\left[\begin{array}{ll}3 & 5 \\ 1 & 2\end{array}\right]$.\\
a) Find $A^{-1}$.\\
b) Find $B^{-1}$.\\
c) Find all $2 \times 2$ matrices $X$ with $A X A^{-1}=B$.

Solution 36 (Inverse of a problem)\\
a) $A^{-1}=\left[\begin{array}{ll}0 & 1 \\ 1 & 0\end{array}\right]$.\\
b) $B^{-1}=\left[\begin{array}{cc}2 & -5 \\ -1 & 3\end{array}\right]$.\\
c) $X=A^{-1} B A=\left[\begin{array}{ll}2 & 1 \\ 5 & 1\end{array}\right]$.


\end{document}