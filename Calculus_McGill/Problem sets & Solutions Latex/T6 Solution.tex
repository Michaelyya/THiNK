\documentclass[10pt]{article}
\usepackage[utf8]{inputenc}
\usepackage[T1]{fontenc}
\usepackage{amsmath}
\usepackage{amsfonts}
\usepackage{amssymb}
\usepackage[version=4]{mhchem}
\usepackage{stmaryrd}

\begin{document}
\begin{enumerate}
  \item For each of the following functions find the first, second and third derivatives $\left(\frac{d f}{d x}, \frac{d^{2} f}{d x^{2}}, \frac{d^{3} f}{d x^{3}}\right)$ :\\
(a) $f(x)=\frac{x^{4}}{24}+\frac{x^{3}}{6}+\frac{x^{2}}{2}+x+1$\\
(b) $f(x)=\sin x$\\
(c) $f(x)=\cos x$
\end{enumerate}

\section*{Solution}
(a)


\begin{align*}
\frac{d f}{d x} & =\frac{x^{3}}{6}+\frac{x^{2}}{2}+x+1  \tag{1}\\
\frac{d^{2} f}{d x^{2}} & =\frac{x^{2}}{2}+x+1  \tag{2}\\
\frac{d^{3} f}{d x^{3}} & =x+1 \tag{3}
\end{align*}


(b)


\begin{align*}
\frac{d f}{d x} & =\cos x  \tag{4}\\
\frac{d^{2} f}{d x^{2}} & =-\sin x  \tag{5}\\
\frac{d^{3} f}{d x^{3}} & =-\cos x \tag{6}
\end{align*}


(c)


\begin{align*}
\frac{d f}{d x} & =-\sin x  \tag{7}\\
\frac{d^{2} f}{d x^{2}} & =-\cos x  \tag{8}\\
\frac{d^{3} f}{d x^{3}} & =\sin x \tag{9}
\end{align*}


\begin{enumerate}
  \setcounter{enumi}{1}
  \item Evaluate the following derivatives:\\
(a) $\frac{d}{d x}\left(a e^{x}+\frac{b}{x}+\frac{c}{x^{2}}\right)$\\
(f) $\frac{d^{2}}{d x^{2}}\left(x^{4} e^{x}\right)$\\
(b) $\frac{d}{d x}\left(\sqrt{x}+\frac{1}{\sqrt[3]{x}}\right)^{2}$\\
(g) $\frac{d}{d x}(x \sec x \tan x)$\\
(c) $\frac{d}{d x}\left(\frac{x-x^{2}}{\sqrt{x}}\right)$\\
(h) $\frac{d}{d x}\left(\frac{a x+b}{c x+d}\right)$\\
(d) $\left.\frac{d}{d x}\left(\left(3 x^{3}+2\right)(7 x+2)\right)\right)$\\
(i) $\frac{d}{d x}\left(\frac{1-\sec x}{\cot x}\right)$\\
(e) $\frac{d}{d x}\left(\sec x e^{x}\right)$\\
(j) $\frac{d}{d x}\left(\frac{x^{3} e^{x}+1}{2 x+e^{x}}\right)$
\end{enumerate}

\section*{Solution}
(a) We can differentiate term by term, remember $a, b, c$ are just constants, and remember we can write $\frac{1}{x}=x^{-1}, \frac{1}{x^{2}}=x^{-2}$. So the derivative is $a e^{x}-\frac{b}{x^{2}}-\frac{2 c}{x^{3}}$\\
(b) We can expand out the brackets to get

$$
\frac{d}{d x}\left(x+2 x^{1 / 6}+x^{-2 / 3}\right)=1+\frac{1}{3} x^{-5 / 6}-\frac{2}{3} x^{-5 / 6}
$$

(c) Again, we can simplify this expression to get

$$
\frac{d}{d x}\left(x^{1 / 2}-x^{3 / 2}\right)=\frac{1}{2} x^{-1 / 2}-\frac{3}{2} x^{1 / 2}
$$

(d) We can expand the brackets, or we can also apply the product rule with $u=$ $\left(3 x^{2}+2\right), v=(7 x+2)$. We know the derivative is equal to

$$
u \frac{d v}{d x}+v \frac{d u}{d x}=\left(3 x^{2}+2\right) \cdot 7+(7 x+2) \cdot(6 x)
$$

(e) Here we apply the product rule with $u=\sec x, v=e^{x}$ and remember that $\frac{d}{d x} \sec x=\sec x \tan x$ so we get

$$
\sec x \tan x e^{x}+\sec x e^{x}
$$

(f) First we will compute the first derivative using the product rule with $u=x^{4}, v=$ $e^{x}$, then we get

$$
4 x^{3} e^{x}+x^{4} e^{x}
$$

we need to differentiate this again to get the second derivative. We will apply the product rule to each of the terms, which gives the final answer

$$
12 x^{2} e^{x}+4 x^{3} e^{x}+4 x^{3} e^{x}+x^{4} e^{x}
$$

(g) Here we will start by applying the product rule with $u=x, v=\sec x \tan x$ to get that the derivative is

$$
\sec x \tan x+x \frac{d}{d x}(\sec x \tan x)
$$

we can apply the product rule again to differentiate $\sec x \tan x$ with $u=\sec x, v=$ $\tan x$ which gives the final answer

$$
\sec x \tan x+x\left(\sec x \tan ^{2} x+\sec ^{3} x\right)
$$

(h) We apply the quotient rule with $u=a x+b, v=c x+d$. Remember the quotient rule is

$$
\frac{d}{d x}\left(\frac{u(x)}{v(x)}\right)=\frac{u^{\prime}(x) v(x)-u(x) v^{\prime}(x)}{v(x)^{2}}
$$

So we get

$$
\frac{a(c x+d)-c(a x+b)}{(c x+d)^{2}}=\frac{a d-b c}{(c x+d)^{2}}
$$

(i) Here again we apply the quotient rule with $u=1-\sec x, v=\cot x$ and remember that $\frac{d}{d x} \cot x=-\csc ^{2} x$ we get

$$
\frac{\sec x \tan x \cot x+\csc ^{2} x(1-\sec x)}{\cot ^{2} x}
$$

(j) Apply the quotient rule with $u=x^{3} e^{x}+1, v=2 x+e^{x}$ gives

$$
\frac{\frac{d}{d x}\left(x^{3} e^{x}\right)\left(2 x+e^{x}\right)-\left(x^{3} e^{x}+1\right)\left(2+e^{x}\right)}{\left(2 x+e^{x}\right)^{2}}
$$

The final thing we need is the derivative of $x^{3} e^{x}$ in the numerator, which we can calculate using the product rule and is equal to $3 x^{2} e^{x}+x^{3} e^{x}$ therefore the final answer isn

$$
\frac{\left(3 x^{2} e^{x}+x^{3} e^{x}\right)\left(2 x+e^{x}\right)-\left(x^{3} e^{x}+1\right)\left(2+e^{x}\right)}{\left(2 x+e^{x}\right)^{2}}
$$

\begin{enumerate}
  \setcounter{enumi}{2}
  \item Find the equations of the tangent line of the curves at the given point:\\
(a) $y=x e^{x}+x^{3}$ at $(0,0)$\\
(b) $y=\sin x+\cos x$ at $(0,1)$\\
(c) $y=e^{x} \cos x+\sin x$ at $(0,1)$
\end{enumerate}

\section*{Solution}
(a) First we evaluate the derivative

$$
\frac{d}{d x}\left(x e^{x}+x^{3}\right)=e^{x}+x e^{x}+3 x^{2}
$$

so at 0 the slope is $y^{\prime}(0)=1$. The line we are looking for has slope 1 and goes through the origin so has equation $y=x$\\
(b) First we evaluate the derivative

$$
\frac{d}{d x}(\sin x+\cos x)=\cos x-\sin x
$$

so at 0 the slope is $y^{\prime}(0)=1$. We are looking for a line with slope 1 and goes through the point $(0,1)$. That is $y=x+1$.\\
(c) First we evaluate the derivative

$$
\frac{d}{d x}\left(e^{x} \cos x+\sin x\right)=e^{x} \cos x-e^{x} \sin x+\cos x
$$

so the slope at 0 is $y^{\prime}(0)=2$. We're looking for a line with slope 2 going through the point $(0,1)$. That is $y=2 x+1$\\
4. (a) Where does $f(x)=e^{x} \cos x$ have a horizontal tangent line?\\
(b) Where does $f(x)=\frac{x^{2}+1}{x+1}$ have a tangent line parallel to the line $2 y=x-3$ ?

\section*{Solution}
(a) We differentiate

$$
f^{\prime}(x)=e^{x} \cos x-e^{x} \sin x
$$

The tangent line is horizontal means it has slope 0 . So we want to solve the equation $f^{\prime}(x)=0$. So this means we want to solve

$$
e^{x} \cos x-e^{x} \sin x=0
$$

Remember $e^{x}$ is never 0 so we can divide by it to get $\sin x=\cos x$. This happens when $x=\pi / 4+n \pi$ for any integer $n$.\\
(b) Differentiate by using the quotient rule

$$
f^{\prime}(x)=\frac{2 x(x+1)-\left(x^{2}+1\right)}{(x+1)^{2}}=\frac{x^{2}+2 x-1}{(x+1)^{2}}
$$

To be parallel to the line $2 y=x-3$ means it has the same slope, which is $\frac{1}{2}$, so we want to solve the equation

$$
\frac{x^{2}+2 x-1}{(x+1)^{2}}=\frac{1}{2}
$$

Moving things around we get

$$
2 x^{2}+4 x-2=x^{2}+2 x+1
$$

which is the same as

$$
x^{2}+2 x-3=0
$$

which has solutions $x=1, x=-2$\\
5. Let

$$
f(x)= \begin{cases}x^{2}, & x \leq 2 \\ m x+b, & x>2\end{cases}
$$

Find the values of $m$ and $b$ that make $f$ differentiable everywhere.

\section*{Solution}
Because $x^{2}, m x+b$ are differentiable, $f(x)$ is differentiable away from $x=2$, we have to make sure it is differentiable at $x=2$. The derivative of $x^{2}$ is $2 x$ so at 2 it is 4 . The derivative of $m x+b$ at 2 is $m$. So we have to make sure the two derivatives agree, which happens when $m=2$.\\
6. Suppose that $f, g, h$ are differentiable functions. Prove that

$$
(f g h)^{\prime}=f^{\prime} g h+f g^{\prime} h+f g h^{\prime}
$$

What can you say about the derivative of a product of $n$ differentiable functions?

\section*{Solution}
We are going to apply the chain rule with $u=(f g), v=h$ we get

$$
(f g h)^{\prime}=(f g)^{\prime} h+(f g) h^{\prime}
$$

Now apply the chain rule for $(f g)^{\prime}$ gives the final answer

$$
f^{\prime} g h+f g^{\prime} h+f g h^{\prime}
$$

\begin{enumerate}
  \setcounter{enumi}{6}
  \item Find the derivative of the function. Do not simplify.\\
(a) $y=\left(x^{2}+7 x+2\right)^{20}$\\
(e) $f(x)=\ln \left(e^{-x}+x e^{-x}\right)$\\
(b) $y=x(4 x+1)^{100}$\\
(f) $h(t)=t \ln \left(\frac{1}{t}\right)$\\
(c) $y=\cos (\sqrt{\sin x})$\\
(g) $f(x)=x^{x}$\\
(d) $y=\tan ^{2}(\sin (3 x+\ln x))$\\
(h) $y=e^{e^{x}}$
\end{enumerate}

\section*{Solution}
(a) We want to apply the chain rule. We will let $f(x)=x^{20}$ and $g(x)=x^{2}+7 x+2$. We know that our function is $f(g(x))$ and its derivative is

$$
f^{\prime}(g(x)) g^{\prime}(x)
$$

In our case $f^{\prime}(x)=20 x^{19}$ and $g^{\prime}(x)=2 x+7$. Therefore the derivative is

$$
20\left(x^{2}+7 x+2\right)^{19} \cdot(2 x+7)
$$

(b) First we apply the product rule. The derivative is

$$
(4 x+1)^{100}+x \cdot \frac{d}{d x}(4 x+1)^{100}
$$

So we want to evaluate $\frac{d}{d x}(4 x+1)^{100}$ for which we will use the chain rule with $f(x)=x^{1} 00$, and $g(x)=4 x+1$. The chain rule tells us that the derivative is

$$
100(4 x+1)^{99} \cdot 4
$$

So the final answer is

$$
(4 x+1)^{100}+400 x(4 x+1)^{99}
$$

(c) Here we want to apply the chain rule with $f(x)=\cos x$ and $g(x)=\sqrt{\sin x}$ the chain rule says that the derivative of $f(g(x))$ is $f^{\prime}(g(x)) g^{\prime}(x)$. We know that $f^{\prime}(x)=-\sin x$, and $g^{\prime}(x)$ is the derivative

$$
\frac{d}{d x} \sqrt{\sin x}
$$

To evaluate $g^{\prime}(x)$ we need to apply the chain rule once more. We will use $u(x)=$ $\sqrt{x}$ and $v(x)=\sin x$ and we want to differentiate $u(v(x))$. So

$$
g^{\prime}(x)=\frac{d}{d x} \sqrt{\sin x}=\cos x \frac{1}{2 \sqrt{\sin x}}
$$

Putting it all together we see that the final answer is

$$
-\sin (\sqrt{\sin x}) \cdot \frac{\cos x}{2 \sqrt{\sin x}}
$$

(d) Similar to the previous question, we will have to apply the chain rule several times in this question. We start by applying the chain rule to the functions $f(x)=\tan ^{2} x, g(x)=\sin (3 x+\ln x)$. We know that the final answer will be

$$
f^{\prime}(g(x)) g^{\prime}(x)
$$

so we need to differentiate $f(x)$ and $g(x)$. Lets start with $f(x)=(\tan x)^{2}$. We will use the chain rule to differentiate it with $s(x)=x^{2}$ and $t(x)=\tan x$. We know that $s^{\prime}(x)=2 x$ and $t^{\prime}(x)=\sec ^{2} x$. Therefore

$$
f^{\prime}(x)=2(\tan x) \sec ^{2} x
$$

Now lets differentiate $g(x)$. Once more we will apply the chain rule, this time with $u(x)=\sin x$ and $v(x)=3 x+\ln x$. We know that $u^{\prime}(x)=\cos x$ and $v^{\prime}(x)=3+\frac{1}{x}$. Therefore

$$
g^{\prime}(x)=\cos (3 x+\ln x) \cdot\left(3+\frac{1}{x}\right)
$$

So we can conclude the the following messy looking final answer

$$
2 \tan (\sin (3 x+\ln x)) \sec ^{2}(\sin (3 x+\ln x)) \cdot \cos (3 x+\ln x) \cdot\left(3+\frac{1}{x}\right)
$$

(e) We will apply the chain rule with $u(x)=\ln x, v(x)=e^{-x}+x e^{-x}$. We know $u^{\prime}(x)=\frac{1}{x}$ and $v^{\prime}(x)=-e^{-x}+e^{-x}-x e^{-x}$, where we used the product rule to find $v^{\prime}(x)$. Therefore, by the chain rule

$$
f^{\prime}(x)=\frac{1}{e^{-x}+x e^{-x}} \cdot\left(-x e^{-x}\right)
$$

(f) By first applying the product rule we can see that

$$
h^{\prime}(t)=\ln \left(\frac{1}{t}\right)+t \frac{d}{d t} \ln \left(\frac{1}{t}\right)
$$

To evaluate $\frac{d}{d t} \ln (1 / t)$ we can apply the chain rule with $f(t)=\ln t$ and $g(t)=\frac{1}{t}$. We know $f^{\prime}(t)=\frac{1}{t}$ and $g^{\prime}(t)=-\frac{1}{t^{2}}$. So

$$
\frac{d}{d t}=t \cdot\left(-\frac{1}{t^{2}}\right)=-\frac{1}{t}
$$

Therefore

$$
h^{\prime}(t)=\ln \left(\frac{1}{t}\right)-1
$$

Another way to do this is to note that $\ln (1 / t)=-\ln (t)$.\\
(g) To do this, we first write

$$
x^{x}=e^{\ln \left(x^{x}\right)}=e^{x \ln x}
$$

Now we can use the chain rule with $u(x)=e^{x}$ and $v(x)=x \ln x$. We know $u^{\prime}(x)=e^{x}$ and $v^{\prime}(x)=\ln x+1$. Therore

$$
f^{\prime}(x)=e^{x \ln x} \cdot(\ln x+1)=x^{x}(\ln x+1)
$$

(h) Here we apply the chain rule with $u(x)=e^{x}$ and $v(x)=e^{x}$, then $u^{\prime}(x)=e^{x}$ and $v^{\prime}(x)=e^{x}$. So

$$
y^{\prime}(x)=e^{e^{x}} \cdot e^{x}
$$


\end{document}