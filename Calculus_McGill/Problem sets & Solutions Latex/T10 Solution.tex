\documentclass[10pt]{article}
\usepackage[utf8]{inputenc}
\usepackage[T1]{fontenc}
\usepackage{amsmath}
\usepackage{amsfonts}
\usepackage{amssymb}
\usepackage[version=4]{mhchem}
\usepackage{stmaryrd}

\begin{document}
\begin{enumerate}
  \item Prove the following identities\\
(a)
\end{enumerate}

$$
\frac{1+\tanh x}{1-\tanh x}=e^{2 x}
$$

(b)

$$
\cosh 2 x=\cosh ^{2} x+\sinh ^{2} x
$$

(c)

$$
(\cosh x+\sinh x)^{n}=\cosh n x+\sinh n x
$$

For any real number $n$.

\section*{Solution}
(a) Remember that

$$
\tanh x=\frac{e^{x}-e^{-x}}{e^{x}+e^{-x}}
$$

We can plug that into the left hand side to get

$$
\frac{1+\frac{e^{x}-e^{-x}}{e^{x}+e^{-x}}}{1-\frac{e^{x}-e^{-x}}{e^{x}+e^{-x}}}=\frac{e^{x}+e^{-x}+e^{x}-e^{-x}}{e^{x}+e^{-x}-e^{x}+e^{-x}}=\frac{2 e^{x}}{2 e^{-x}}=e^{2 x}
$$

(b) We can plug in the definitions of cosh and sinh in the right hand side to get

$$
\left(\frac{e^{x}+e^{-x}}{2}\right)^{2}+\left(\frac{e^{x}-e^{-x}}{2}\right)^{2}=\frac{1}{4}\left[e^{2 x}+2+e^{-2 x}+e^{2 x}-2+e^{-2 x}\right]=\frac{e^{2 x}+e^{-2 x}}{2}
$$

and by definition

$$
\cosh 2 x=\frac{e^{2 x}+e^{-2 x}}{2}
$$

(c) For this one note that

$$
\cosh x+\sinh x=e^{x}
$$

Therefore

$$
(\cosh x+\sinh x)^{n}=e^{n x}
$$

Similarly

$$
\cosh n x+\sinh n x=e^{n x}
$$

\begin{enumerate}
  \setcounter{enumi}{1}
  \item Find the derivative of the function.\\
(a) $y=\tan ^{-1}(\sinh x)$\\
(b) $y=\tanh ^{-1}\left(x^{3}\right)$\\
(c) $y=\operatorname{sech}(\tanh x)$
\end{enumerate}

\section*{Solution}
(a) We will apply the chain rule. Recall that

$$
\begin{gathered}
\frac{d}{d x} \tan ^{-1} x=\frac{1}{1+x^{2}} \\
\frac{d}{d x} \sinh x=\cosh x
\end{gathered}
$$

Therefore

$$
\frac{d}{d x} \tanh ^{-1}(\sinh x)=\frac{\cosh x}{1+\sinh ^{2} x}=\frac{\cosh x}{\cosh ^{2} x}=\frac{1}{\cosh x}
$$

(b) We will apply the chain rule. Recall that

$$
\frac{d}{d x} \tanh ^{-1} x=\frac{1}{1-x^{2}}
$$

Therefore

$$
\frac{d}{d x} \tanh ^{-1}\left(x^{3}\right)=\frac{3 x^{2}}{1-x^{6}}
$$

(c) We have $\operatorname{sech} x=\frac{1}{\cosh x}$ therefore

$$
\frac{d}{d x} \operatorname{sech} x=-\frac{\sinh x}{\cosh ^{2} x}=-\tanh x \operatorname{sech} x
$$

Therefore

$$
\frac{d}{d x} \operatorname{sech}(\tanh x)=-\tanh (\tanh x) \operatorname{sech}(\tanh x) \cdot \frac{1}{1-x^{2}}
$$

\begin{enumerate}
  \setcounter{enumi}{2}
  \item Verify that $f(x)=\sqrt{x}-\frac{1}{3} x$ satisfies the hypotheses of Rolle's Theorem on the interval $[0,9]$. Find all numbers $c$ satisfying the conclusion of the theorem.
\end{enumerate}

\section*{Solution}
First we need to check that $f$ satisfies the hypothesis of Rolle's Theorem

\begin{itemize}
  \item $f$ is composed of elementary functions whose domain contains the interval $[0,9]$ therefore, $f$ is continuous on the interval.
  \item $f$ is differentiable on the interval $(0,9)$ with derivative
\end{itemize}

$$
f^{\prime}(x)=\frac{1}{2 \sqrt{x}}-\frac{1}{3}
$$

Note that $f$ is not differentiable at 0 , but that is outside of the interval $(0,9)$.

\begin{itemize}
  \item $f(0)=f(9)=0$.
\end{itemize}

So $f$ satisfies the hypothesis of Rolle's Theorem. Therefore we can conclude that there exists some number $c$ between 0 and 9 such that $f^{\prime}(c)=0$. This means

$$
\frac{1}{2 \sqrt{c}}-\frac{1}{3}=0
$$

We can solve this equation to find that

$$
c=\frac{9}{4} .
$$

\begin{enumerate}
  \setcounter{enumi}{3}
  \item Explain why, if the graph of a polynomial function has three $x$-intercepts, then it must have at least two points at which its tangent line is horizontal. Is this true for any function having three $x$-intercepts?
\end{enumerate}

\section*{Solution}
Every polynomial is an elementary function and is continuous and differentiable, and its derivative is a polynomial of smaller degree. If $f$ is a polynomial that has three $x$-intercepts, this means there exists three numbers, $x_{1}, x_{2}, x_{3}$ such that

$$
f\left(x_{1}\right)=f\left(x_{2}\right)=f\left(x_{3}\right)=0 .
$$

We can now apply Rolle's theorem to the interval $\left[x_{1}, x_{2}\right]$, which shows that there is a number $c_{1} \in\left(x_{1}, x_{2}\right)$, such that

$$
f^{\prime}\left(c_{1}\right)=0
$$

We can also do the same for the interval $\left[x_{2}, x_{3}\right]$ to get another number $c_{2} \in\left(x_{2}, x_{3}\right)$ such that

$$
f^{\prime}\left(c_{2}\right)=0
$$

This means that the tangent line of the graph of $f$ is horizontal at $c_{1}$ and $c_{2}$.\\
This will be also true for any other function, which is not necessarily a polynomial, as long as it satisfies the hypothesis to Rolle's theorem.\\
5. Show that the equation $x^{4}+4 x+c=0$ has at most two solutions which are real numbers.

\section*{Solution}
We will solve this problem in two steps. First, for the function $f(x)=x^{4}+4 x+c$, we will show that $f^{\prime}(x)$ is equal to 0 exactly once. Then, we will use the previous problem show that if $f(x)=0$ has more than two solutions, then $f^{\prime}(x)=0$ must have at least 2 solutions. This means that $f(x)=0$ cannot have more than two solutions.\\
For the first step, differentiate

$$
f^{\prime}(x)=4 x^{3}+4
$$

So $f^{\prime}(x)=0$, means $x^{3}+1=0$. This equation has only one solution, $x=-1$.\\
We now know that $f(x)$ has only one point with horizontal tangent line (at $x=-1$ ). If $f(x)$ had three $x$-intercepts, then this means it has at least two points with horizontal tangent lines ( which we know is false). Therefore, $f$ can't have three $x$-intercepts. So it has at most two $x$-intercepts.\\
6. Verify that the function satisfies the hypotheses of the MVT on the given interval. Then, find all values of $c$ that satisfy the conclusion of the MVT.\\
(a) $f(x)=\frac{x}{x+2}$ on $[1,4]$\\
(b) $f(x)=e^{-2 x}$ on $[0,3]$

\section*{Solution}
(a) $f(x)$ is composed of elementary functions and is defined everywhere except for $x=-2$, therefore it is continuous on the interval $[1,4]$. We can also differentiate

$$
f^{\prime}(x)=\frac{2}{(x+2)^{2}}
$$

which is again defined everywhere except for $x=-2$, so $f$ is differentiable on $[1,4]$. We have shown that $f$ satisfies the hypothesis of the MVT. Therefore we can conclude that there exists a number $c \in(1,4)$ such that

$$
f^{\prime}(c)=\frac{f(4)-f(1)}{4-1}=\frac{1}{9}
$$

This means that

$$
\frac{2}{(c+2)^{2}}=\frac{1}{9}
$$

Therefore

$$
c=3 \sqrt{2}-2
$$

(b) Same as before $f(x)$ is composed of elementary functions and its domain is all of the real numbers, so it is certainly continuous on the interval $[0,3]$ and its derivative is

$$
f^{\prime}(x)=-2 e^{-2 x}
$$

which is defined on the interval $[0,3]$. So we can conclude that $f$ satisfies the hypothesis of the MVT and there exists some $c \in(0,3)$ such that

$$
f^{\prime}(c)=\frac{f(3)-f(1)}{3-1}=\frac{e^{-6}-e^{-2}}{2}
$$

Therefore,

$$
c=-2 \ln \left(\frac{e^{-2}-e^{-6}}{4}\right)
$$

\begin{enumerate}
  \setcounter{enumi}{6}
  \item A number $a$ is called a fixed point of a function $f$ if $f(a)=a$. Prove that if $f$ satisfies the hypothesis of the MVT and $f^{\prime}(x) \neq 1$ for all $x$, then $f$ has at most one fixed point. Hint: Suppose there are at least two fixed points, call them $a$ and $b, a \neq b$. Then, use the MVT.
\end{enumerate}

\section*{Solution}
If $f$ has two fixed points $a$ and $b$ (So $f(a)=a$ and $f(b)=b$ ), and $f$ also satisfies the hypothesis of the MVT, then we know there exists some number $c \in(a, b)$ such that

$$
f^{\prime}(c)=\frac{f(b)-f(a)}{b-a}=\frac{b-a}{b-a}=1
$$

but we have assumed that $f^{\prime}(x) \neq 1$. So such a number $c$ cannot exist. Therefore, we can't have had two fixed points to start with, otherwise we will have a contradiction.


\end{document}