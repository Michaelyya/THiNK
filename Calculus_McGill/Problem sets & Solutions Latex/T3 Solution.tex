\documentclass[10pt]{article}
\usepackage[utf8]{inputenc}
\usepackage[T1]{fontenc}
\usepackage{amsmath}
\usepackage{amsfonts}
\usepackage{amssymb}
\usepackage[version=4]{mhchem}
\usepackage{stmaryrd}

\begin{document}
\begin{enumerate}
  \item Which of the following functions is continuous on the given intervals?\\
(a) $g(t)=\frac{t^{2}+5 t}{2 t+1},[0, \infty)$\\
(c) $f(x)=\left\{\begin{array}{ll}\frac{x-3}{x^{2}-9} & x \neq 3 \\ \frac{1}{6} & x=3\end{array},[0, \infty)\right.$\\
(b) $f(x)=\left\{\begin{array}{ll}\frac{x-3}{x^{2}-9} & x \neq 3 \\ 0 & x=3\end{array},[0, \infty)\right.$\\
(d) $f(x)=\left\{\begin{array}{ll}\sin (x) & x<\pi / 4 \\ \cos (x) & x \geq \pi / 4\end{array},[0, \infty)\right.$
\end{enumerate}

\section*{Solution}
(a) The function $\frac{t^{2}+5 t}{2 t+1}$ is continuous everywhere except where it has a vertical asymptote at $t=-\frac{1}{2}$, but $-\frac{1}{2}$ is not in the interval specified, therefore the function is continuous on the interval $[0, \infty)$\\
(b) This function is continuous everywhere except possibly at $x=3$. Notice that $\frac{x-3}{x-9}=\frac{1}{x+3}$. So $\lim _{x \rightarrow 3} \frac{x-3}{x-9}=\frac{1}{6}$. So the function is not continuous at 3 since $f(3)=0 \neq \frac{1}{6}$.\\
(c) This function is continuous by the same argument from part 2.\\
(d) Again this function is continuous everywhere except possibly at $\pi / 4$. By checking the limit from the right and from the left at $\pi / 4$, we see that they agree since $\sin (\pi / 4)=\cos (\pi / 4)$.\\
2. Evaluate the following limits if the exist:\\
(a) $\lim _{x \rightarrow \pi} \sin (x+\sin x)$\\
(d) $\lim _{x \rightarrow 0} \frac{\sqrt{3+x}-\sqrt{3}}{x}$\\
(b) $\lim _{x \rightarrow 0} x \sin \left(\frac{\pi}{x^{2}}\right)+\cos x$\\
(e) $\lim _{x \rightarrow-1} \frac{x^{2}-4 x}{x^{2}-3 x-4}$\\
(c) $\lim _{t \rightarrow 0} \frac{1}{t \sqrt{t+1}}-\frac{1}{t}$\\
(f) $\lim _{x \rightarrow 2} \arctan \left(\frac{x^{2}-4}{3 x^{2}-6 x}\right)$

\section*{Solution}
(a) $\sin x$ is continuous therefore

$$
\lim _{x \rightarrow \pi} \sin (x+\sin x)=\sin \left(\lim _{x \rightarrow \pi}(x+\sin x)\right)
$$

and since $x$ and $\sin x$ are continuous functions we can just plug $\pi$ in to get

$$
\sin (\pi+\sin (\pi))=0
$$

(b) We know that

$$
-1 \leq \sin \left(\frac{\pi}{x^{2}}\right) \leq 1
$$

therefore

$$
-x+\cos x \leq x \sin \left(\frac{\pi}{x^{2}}\right)+\cos x \leq x+\cos x
$$

The left and right parts of the theorem both tend to 1 therefore by the squeeze theorem

$$
\lim _{x \rightarrow 0} x \sin \left(\frac{\pi}{x^{2}}\right)+\cos x=1
$$

(c) By taking $\frac{1}{t}$ as a common factor we can rewrite the expression as

$$
\frac{1-\sqrt{t+1}}{t \sqrt{t+1}}
$$

Now we multiply the top and the bottom by $1+\sqrt{t+1}$ to get

$$
\frac{-t}{t \sqrt{t+1}(1+\sqrt{t+1})}=\frac{-1}{\sqrt{t+1}(1+\sqrt{t+1})}
$$

that final expression is continuous at 0 and has 0 in its domain so the limit is equal to $\frac{-1}{2}$\\
(d) We do a similar trick to the one we did in part (c) which is to rationalize one of the expressions by multiplying by $\sqrt{3+x}+\sqrt{3}$ at the top and bottom to get

$$
\frac{x}{\sqrt{3+x}+\sqrt{3}}=\frac{1}{\sqrt{3+x}+\sqrt{3}}
$$

and again the last expression clearly has 0 in its domain and is continuous, so the limit tends to $\frac{1}{2 \sqrt{3}}$\\
(e) We begin by factorizing the numerator and denominator

$$
\frac{x(x-4)}{(x-4)(x+1)}=\frac{x}{x+1}
$$

so we can see that from the right the limit tends to $+\infty$ and from the left it tends to $-\infty$ therefore the limit does not exist.\\
(f) $\arctan$ is a continuous function therefore this limit is equal to

$$
\arctan \left(\lim _{x \rightarrow 2} \frac{x^{2}-4}{3 x^{2}-6 x}\right)
$$

we can again factorize the fraction and simplify to get

$$
\frac{x+2}{3 x}
$$

which has limit $\frac{2}{3}$ as $x$ tends to 2 , therefore the final limit is $\operatorname{aarctan}\left(\frac{2}{3}\right)$\\
3. Determine if the statements below are true or false. If true, justify your answer; if not, provide a counterexample.\\
(a) If $f(x)$ is continuous on the interval $(0,1)$, then $f(x)^{2}$ is also continuous on the same interval.\\
(b) If $f(x)^{2}$ is continuous on the interval $(0,1)$, then $f(x)$ is also continuous on the same interval.\\
(c) If $f(x)$ and $g(x)$ are both continuous on the interval $(0,1)$, then $f(x)+g(x)$ is continuous on the same interval.\\
(d) If $f(x)+g(x)$ is continuous on the interval $(0,1)$, then so are $f(x)$ and $g(x)$ on the same interval.\\
(e) If both $f(x)+g(x)$ and $f(x)-g(x)$ are continuous on $(0,1)$, then so are $f(x)$ and $g(x)$ on the same interval.

\section*{Solution}
(a) Yes, this is true by the composition law for continuity, since $f(x)^{2}$ is the result of composing $g(x)=x^{2}$ with $f(x)$ i.e

$$
f(x)^{2}=g(f(x))
$$

and since both $g(x)$ and $f(x)$ are continuous then $g(f(x))$ is also continuous.\\
(b) This is not true, consider the function

$$
f(x)= \begin{cases}-1 & x \leq \frac{1}{2} \\ 1 & x>\frac{1}{2}\end{cases}
$$

This function is not continuous at $\frac{1}{2}$ but $f(x)^{2}=1$ on the interval $(0,1)$ which is continuous, hence we have come up with a counterexample.\\
(c) This is true, and is one of our rules for continuous functions. The sum of two continuous functions is always continuous.\\
(d) This is not true. Consider

$$
f(x)= \begin{cases}-1 & x \leq \frac{1}{2} \\ 1 & x>\frac{1}{2}\end{cases}
$$

and

$$
g(x)= \begin{cases}1 & x \leq \frac{1}{2} \\ -1 & x>\frac{1}{2}\end{cases}
$$

Then similarly to part (b), neither $f$ or $g$ are continuous at $\frac{1}{2}$ but $f(x)+g(x)=0$, therefore $f(x)+g(x)$ is continuous.\\
(e) This is true, notice that

$$
f(x)=\frac{(f(x)+g(x))+(f(x)-g(x))}{2}
$$

so since $f+g$ and $f-g$ are both continuous, then their sum is clearly continuous, and by dividing by 2 we get another continuous function by applying our rules for continuous functions. We can repeat the same argument for $g(x)$ by noting

$$
f(x)=\frac{(f(x)+g(x))-(f(x)-g(x))}{2}
$$

\begin{enumerate}
  \setcounter{enumi}{3}
  \item Use the Intermediate Value Theorem to show that the following equations have a root on the given intervals\\
(a) $\cos x=x$ on $(0,1)$\\
(b) $\ln x=e^{-x}$ on (1,2)
\end{enumerate}

\section*{Solution}
(a) We will apply the Intermediate Value Theorem to the function $f(x)=\cos x-x$. Notice $f(0)=1>0$ and $f(1)=\cos (1)-1<0$. So by the IVT there is a number $c$ between 0 and 1 such that $f(c)=0$, and that means that $\cos (c)=c$.\\
(b) We do the exact thing as above with the function $g(x)=\ln x-e^{-x}$. Notice $g(1)=-\frac{1}{e}<0$ and $g(2)=\ln (2)-\frac{1}{e^{2}}>0$. Therefore, there is a number $c$ between 1 and 2 such that $g(c)=0$.\\
5. In this question we will use the Intermediate Value Theorem to approximate a solution to the equation

$$
x=e^{x-2}
$$

(a) First, show that this equation has a solution in the interval $(0,1)$.\\
(b) Now split this interval into two halves $\left(0, \frac{1}{2}\right)$ and $\left(\frac{1}{2}, 1\right)$. Show that one of these two intervals has a solution to our equation. Which one is it?\\
(c) Split the interval you got in part (b) into two halves again to show that one of the four intervals $\left(0, \frac{1}{4}\right),\left(\frac{1}{4}, \frac{1}{2}\right),\left(\frac{1}{2}, \frac{3}{4}\right),\left(\frac{3}{4}, 1\right)$ has a solution to our equation. Which one is it?\\
(d) Repeat this process of splitting the intervals in halves until you have narrowed down the solution to 2 decimal places.

\section*{Solution}
We consider the function $f(x)=x-e^{x-2}$. Notice $f(0)=-e^{-2}<0$ and $f(1)=$ $1-e^{-1}>0$ therefore there's a root between 0 and 1 . Then repeat this process: $f\left(\frac{1}{2}\right)>0$ therefore there's a root betwee 0 and $\frac{1}{2}$, now we investigate the halfway point $f\left(\frac{1}{4}\right)>0$ so we can conclude there's a root between 0 and $\frac{1}{4}$. We keep going: The midpoint is now $\frac{1}{8}$ and $f\left(\frac{1}{8}\right)<0$ so there's a root between $\frac{1}{8}$ and $\frac{1}{4}$. The midpoint between them is $\frac{3}{16}$ and $f\left(\frac{3}{16}\right)>0$ so there's a root between $\frac{1}{8}$ and $\frac{3}{16}$. The midpoint now is $\frac{5}{32}$ and $f\left(\frac{5}{32}\right)<0$ so there's a root between $\frac{5}{32}$ and $\frac{3}{16}$. Narrowing down, there's a root between $\frac{5}{32}$ and $\frac{21}{128}$ and now we've narrowed it down to an interval of length less than 0.01.\\
6. A mountain climber starts climbing a mountain at 10 am and they reach the top at 2 pm where they stay for the night. The next day they begin their descent at 10am and reach the bottom at 2 pm , they take the exact same path they took the day before. Is there necessarily a time between 10 am and 2 pm at which the climber was at the exact same spot on both days? (Note the the climber doesn't necessarily ascend and descend at the same speed)

\section*{Solution}
Let $f(x)$ be the function describing the position of the climber during ascent, and $g(x)$ be the function describing the position during descent. Then at 10am the climber is at\\
the bottom on the first day and at the top on the second day, therefore $f(10)-g(10)<$ 0 , at 2 pm it's switched, so $f(14)-g(14)>0$. Since the climber moves continuously, the functions $f$ and $g$ are continuous, therefore $f(x)-g(x)$ is continuous, therefore there is a time $c$ between 8 and 14 such that $f(c)=g(c)$.\\
7. Evaluate the following limits if they exist\\
(a) $\lim _{x \rightarrow+\infty} \cos x$\\
(d) $\lim _{x \rightarrow-\infty} \frac{x}{x^{2}+1}$\\
(b) $\lim _{x \rightarrow+\infty} \frac{x+1}{x}$\\
(e) $\lim _{x \rightarrow+\infty} \frac{x^{3}-3 x^{2}+3 x-1}{3 x^{3}+27 x^{2}+9 x+1}$\\
(c) $\lim _{x \rightarrow+\infty} \frac{x}{x+1}$\\
(f) $\lim _{x \rightarrow 1^{+}} \arctan \left(\frac{1}{x-1}\right)$

\section*{Solution}
(a) The function $\cos x$ alternates between -1 and 1 therefore the limit does not exist.\\
(b) We can simplify this function to $1+\frac{1}{x}$. As $x$ gets bigger $\frac{1}{x}$ gets much smaller and tends to 0 , therefore $1+\frac{1}{x}$ tends to 1 .\\
(c) We divide at the top and bottom by $x$ to get

$$
\frac{1}{1+1 / x}
$$

As in the previous part $\frac{1}{x}$ tends to 0 therefore the whole expression tends to $\frac{1}{1+0}=1$.\\
(d) Again we divide at the top and bottom by $x x$ to get

$$
\frac{1}{x+\frac{1}{x}}
$$

again we note that $\frac{1}{x}$ tends to 0 and $x$ tends to $\infty$ therefore $x+\frac{1}{x}$ tends to $\infty$ and so $\frac{1}{x+\frac{1}{x}}$ tends to 0 .\\
(e) Here again we divide by the higher power of $x$ in the numerator which is $x^{3}$ to get

$$
\frac{1-\frac{3}{x}+\frac{3}{x^{2}}-\frac{1}{x^{3}}}{3+\frac{27}{x}+\frac{9}{x^{2}}+\frac{1}{x^{3}}}
$$

Notice that all the fractions $\frac{\alpha}{x^{i}}$ will just tend to 0 as $x$ gets bigger. Therefore the numerator tends to 1 and the denominator tends to 3 so the ratio tends to $\frac{1}{3}$\\
(f) As $x$ tends to 1 from the right $\frac{1}{x-1}$ tends to $+\infty$ so this limit is equal to

$$
\lim _{t \rightarrow+\infty} \arctan (t)
$$

which we know is equal to $\pi / 2$.\\
8. Find a rational function $f(x)$ such that $\lim _{x \rightarrow \infty} f(x)=\lim _{x \rightarrow-\infty} f(x)$. Find another rational function $g(x)$ such that $\lim _{x \rightarrow \infty} g(x) \neq \lim _{x \rightarrow-\infty} g(x)$.\\
Reminder: a rational function is a polynomial divided by a polynomial.


\end{document}