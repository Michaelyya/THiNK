\documentclass{article}
\usepackage{enumerate}

\title{Fall 2018 14.01 Problem Set 9 - Solutions}
\date{}
\begin{document}
\maketitle

\section*{Problem 1: True or False (20 points)}
Determine whether the following statements are True or False. Explain your answer.
\begin{enumerate}[(a)]
\item (5 points) Suppose the cost of making a car is cheapest in Japan. Then Japan should specialize in producing cars.

\textbf{Solution:} False. If Japan can make cars more cheaply than any other country in the world, then Japan has an absolute advantage in making cars. However, gains from trade happen from countries specializing in their areas of comparative advantage, and absolute advantage does not imply comparative advantage.

\item (5 points) In order for an IRA to encourage savings (relative to a regular savings account), the substitution effect of higher interest rates must dominate the income effect.

\textbf{Solution:} True. IRAs offer higher returns to savings relative to regular savings account. For higher interest rates to increase savings, the substitution effect must dominate the income effect.

\item (5 points) An index fund is a portfolio of stocks that tracks a broader index such as the Dow Jones Industrial Average or the S\&P 500. Investing in an index fund is better than investing in the stock of an individual company because the index fund always has higher returns.

\textbf{Solution:} False. There is no guarantee that an index fund offers higher returns than the individual stock of a company (consider Amazon stock vs. the DJIA over the past 15 years). However, index funds offer the benefit of diversification, which individual stocks do not.

\item Consider the effect of interest rates on consumption today. Increasing interest rates always has a negative substitution effect (decreases consumption today) and a positive income effect (increases consumption today).

\textbf{Solution:} False. The income effect is ambiguous and depends on whether the person is a borrower or a saver.

\end{enumerate}

\section*{Problem 2: Trade and Production Possibilities Frontier (20 points)}
Consider the production of wine and cheese in France and Spain. This table gives the number of necessary hours to produce each (labor is the only input):

\begin{tabular}{c|c|c|}
& France & Spain \\
\hline
1 Kilo of Cheese & 4 & 6 \\
1 Bottle of wine & 6 & 12 \\
\end{tabular}
