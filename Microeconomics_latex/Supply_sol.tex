\documentclass{article}
\usepackage[utf8]{inputenc}
\usepackage{amsmath}

\title{Fall 2018 14.01 Problem Set 1-Solutions}
\author{}
\date{}
\begin{{document}}
\maketitle

\section{Positive vs. Normative Statements (16 points)}
Identify whether each of the following statements is positive or normative. Briefly justify your answer.

\begin{enumerate}
\item (4 points) The government has a duty to provide basic healthcare and education to every citizen.
\item (4 points) The cost of health insurance is too high.
\item (4 points) The median earnings of a full-time worker with a college degree are almost twice as high as those of a high-school graduate with no college education.
\item (4 points) The current unemployment rate is 3.9\%, the lowest it has been since December 2000.
\end{enumerate}

\textbf{Solution:} The first statement is normative as it provides an opinion about what the role of the government should be. The second statement is also normative since saying that it is “too high” is implicitly expressing that it should be lower. The last two statements are positive as they are just describing facts about the US economy.

\section{True or False (20 points)} 

For each of the following statements, indicate if they are True or False. Justify your answer.

\begin{enumerate}
\item Bill is a football coach. He evaluates his players based on three criteria: height, strength, and speed. Bill prefers one player over another if he is better in at least two of these criteria. Assume that there are no players with the exact same height, nor the exact same strength, nor the exact same speed.
\begin{enumerate}
\item (4 points) Bill’s preferences are complete.
\item (4 points) Bill’s preferences are transitive.
\end{enumerate}
\item (4 points) Ann and Bob are utility maximizing consumers. Given their income and market prices, Ann chooses a bundle that gives her a utility $U_{\text{Ann}} = 100$, while Bob chooses a bundle that gives him a utility $U_{\text{Bob}} = 110$. Therefore, we know that Bob is happier than Ann.
% Rest of the document content is omitted
\end{document}