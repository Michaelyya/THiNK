\documentclass{article}

\begin{document}

\title{Fall 2018 14.01 Problem Set 8 - Solutions}
\maketitle

\section*{Problem 1: True or False (20 points)}
Determine whether the following statements are True or False. Explain your answer.

\begin{enumerate}
\item \emph{(4 points)} Raising the minimum wage may improve the welfare of workers, but will always lead to an increase in the deadweight loss.

\textbf{Solution:} False, if employers have market power, then raising the minimum wage could decrease the DWL.

\item \emph{(4 points)} An increase in the interest rate has an ambiguous effect on the savings of a utility maximizing household.

\textbf{Solution:} True, depending on how strong the income and substitution effects are, savings could either increase or decrease when the interest rate increases.

\item \emph{(4 points)} A student starting a 3 year long post-doc at MIT is considering two alternative car rental programs to use during those 3 years. Program 1 charges a \$1200 initial membership fee and then \$120 per year. Program 2 charges a \$360 initial membership fee and then \$240 per year. The student will always strictly prefer program 2. Assume $r > 0$.

\textbf{Solution:} True. The PDV of the student's expenses under program 1 will be larger as long as:
\begin{equation*}
\frac{1200 + 120 + \frac{120}{1 + r}}{(1 + r)^2} > \frac{360 + 240 + \frac{240}{1 + r}}{(1 + r)^2}
\end{equation*}
which simplifies to $6 > 2 + r$ or $(2r + 1)(3r + 4) > 0$, which is always the case for $(1 + r)^2$ when $r > 0$.

\item \emph{(4 points)} Allen is working, consuming, and saving rationally for retirement. A rise in real interest rates will definitely lead Allen to optimally decide to save more for retirement and consume less today.

\textbf{Solution:} False. The interest rate reflects the trade-off between consumption now and consumption later. So a rise in interest rates does not change the amount that is feasible to consume today, but allows us to consume more later if we save. However, this does not mean we necessarily consume more later.

\item \emph{(4 points)} Suppose that interest rates are at 2 percent and a firm is considering a project with strictly positive net present value. If interest rates increase to 4 percent, the firm will still decide to make the investment to start that project.

\textbf{Solution:} False or Uncertain. This depends on the how the costs and earnings are distributed over the time horizon. For example, a one-time cost that doesn’t yield benefits until 10 years in the future could have a positive NPV when interest rates are 2 percent but a negative NPV when interest rates are 4 percent.
\end{enumerate}

\section*{Problem 2: Monopsony and the labor market (30 points)}

Suppose that a logging company in Northern Carolina faces a perfectly competitive market for the lumber it produces (that is the company takes the price of lumber $p$ as given). However, the logging company is the only employer in the area and has a monopsony.

\begin{enumerate}
\item \emph{(8 points)} Suppose that workers in the area (employed by the logging company) are identical and have utility over consumption and labor given by $u(c, \ell) = c^{5/2} - 3\ell^{3/2}$ and earn $w$ for each unit of labor supplied. The price of one unit of consumption is equal to 1, and the only income a worker has is his labor income. Find the amount of labor that each worker will supply as a function of $w$.

\textbf{Solution:} The worker’s budget constraint is given by $c = w\ell$. Then, the worker’s utility maximization problem becomes \[ \max_{\ell} (w\ell)^{5/2} - 3\ell^{3/2} \] Taking the derivative with respect to $\ell$ yields $\ell = (w/10)^{2/5}$.
\end{enumerate}

% The document continues with other problems...
\end{document}
