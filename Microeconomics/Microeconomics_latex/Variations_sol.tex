\documentclass[11pt]{article}

\title{Fall 2018 14.01 Problem Set 6 - Solutions}
\date{}

\begin{document}

\maketitle

\section*{Problem 1: True/False/Uncertain (20 points)}

\paragraph{1.} \textbf{(4 points)} In a two-player game, a Nash equilibrium is the outcome that maximizes the sum of the players’ payoffs.

\textbf{Solution:} False. Consider the prisoner’s dilemma. The Nash equilibrium does not maximize utility.

\paragraph{2.} \textbf{(4 points)} In a Nash equilibrium in a two-player game, both players must have selected a dominant strategy.

\textbf{Solution:} False. A dominant strategy is one that is a best response no matter what the other player does. A Nash equilibrium is one in which both player play the optimal strategy, given the other player’s strategy. If both players are playing a dominant strategy, this must be a Nash Equilibrium, but not vice-versa.

\paragraph{3.} \textbf{(4 points)} Repeatedly playing the Prisoner’s Dilemma may or may not result in a cooperative solution.

\textbf{Solution:} True. Whether or not cooperation is sustainable will depend on the time horizon, the relative benefit to cooperation, and how much the players value the future.

\paragraph{4.} \textbf{(4 points)} In the models of oligopoly considered in class, the equilibrium price will be strictly lower if there are n + 1 firms than if there are n firms.

\textbf{Solution:} False, in the Bertrand model, with two firms the price will be the marginal cost. If we keep increasing the number of firms, the price will remain at marginal cost.

\paragraph{5.} \textbf{(4 points)} In the models of oligopoly considered in class, consumers are no better off than in a perfectly competitive market.

\textbf{Solution:} True, both in the Cournot and Bertrand models the equilibrium price will be either marginal cost (in Bertrand) or above marginal cost (in Cournot), so the consumers can’t be better off than in the competitive market.

.

.

.

\end{document}