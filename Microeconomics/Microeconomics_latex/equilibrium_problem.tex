\documentclass{article}
\begin{document}
\title{Fall 2018 14.01 Problem Set 4}
\maketitle

\section{Problem 1: True or False (24 points)}

\begin{enumerate}
  \item (4 points) In the short and long run, a profit-maximizing firm will choose its input mix based on $MRTS = -\frac{w}{r}$.
  \item (4 points) Long-run marginal costs can be lower or higher than short-run marginal costs, while long-run average costs can’t be higher than the short-run average costs.
  \item (4 points) In a perfectly competitive market with identical firms, a permanent positive demand shock leads to a permanent increase in the price in the long run.
  \item (4 points) In a perfectly competitive industry, a profit-maximizing firm sets its price equal to its marginal cost in a range where the marginal cost is decreasing.
  \item (4 points) Adding up the individual supply curves $P = 5 + Q_1$ and $P = 3 + Q_2$ will lead to the market supply curve $P = 8 + 2Q$.
  \item (4 points) In 1998, the Kenyan government confiscated and burnt 12 tons of elephant ivory in a gesture to persuade the world to halt ivory trade. The equilibrium quantity in the market for ivory will surely decrease, while the effect on price is ambiguous: it may decrease if this gesture is effective in convincing consumers to stop buying ivory, and will increase otherwise.
\end{enumerate}


\section{Problem 2: Short-run and Long-run equilibrium (26 points)}

Consider a market for skateboards that is in a long-run equilibrium. In this equilibrium, each firm’s short-run and long-run total cost functions are given by:

$SRTC(q) = q^3 - 3q^2 + 3q + 4$

$LRTC(q) = 3q$

The market demand for skateboards is given by $QD(P) = 27 - P$.

\begin{enumerate}
  \item (4 points) What is the equilibrium price in the initial long-run equilibrium?
  \item (4 points) Knowing that cost curves are defined by the above functions, explain why you can infer the number of skateboards each firm produces in long run equilibrium. Calculate the quantity. [Hint: If the market is in a long-run equilibrium, it is also in a short-run equilibrium.]
  % ... Continue like this for all questions. Ellipsis ... represents that.
\end{enumerate}

% Similarly for other problems

\end{document}