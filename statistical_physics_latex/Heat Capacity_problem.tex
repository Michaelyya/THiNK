\documentclass[12pt]{article}

\usepackage{amsmath}
\usepackage{geometry}

\geometry{a4paper}

\title{\vspace{-3.0cm}MASSACHUSETTS INSTITUTE OF TECHNOLOGY\\
Physics Department\\
8.044 Statistical Physics I Spring Term 2013\\
Problem Set \#4}
\date{Due in hand-in box by 12:40 PM, Wednesday, March 6}
\begin{document}
\maketitle

\section*{Problem 1: Heat Capacity at Constant Pressure in a Simple Fluid}
For a simple fluid show that $C_P=(\partial U/\partial T)_P+\beta V_P$. Since the thermal expansion coefficient $\beta$ can be either positive or negative, $C_P$ could be either less than or greater than $(\partial U/\partial T)_P$.

\textit{Hint: Use the first law to find an expression for $dQ$, then expand in terms of the variables T and P.}

\bigskip

\section*{Problem 2: Heat Supplied to a Gas}
An ideal gas for which $C_V=5Nk$ is taken from point a to point b in the figure along three paths: acb, adb, and ab. Here $P_2=2P_1$ and $V_2=2V_1$. Assume that $(\partial U/\partial V)_T=0$

\begin{enumerate}
\item[a)] Compute the heat supplied to the gas (in terms of N, k, and T1) in each of the three processes. 

\textit{Hint: You may wish to find $C_P$ first.}

\item[b)] What is the ``heat capacity" of the gas for the process ab?
\end{enumerate}

\bigskip

\section*{Problem 3: Thermodynamics of a Curie Law Paramagnet}
Simple magnetic systems can be described by two independent variables. State variables of interest include the magnetic field $H$, the magnetization $M$, the temperature $T$, and the internal energy $U$. Four quantities that are often measured experimentally are...

\end{document}