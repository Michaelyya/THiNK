\documentclass{article}
\usepackage{amsmath, amsfonts, amssymb}

\begin{document}
\title{MASSACHUSETTS INSTITUTE OF TECHNOLOGY \\
Physics Department \\
8.044 Statistical Physics I Spring Term 2013 \\
Problem Set \#3}
\date{Due in hand-in box by 12:40 PM, Wednesday, February 27}
\maketitle

\noindent
\textbf{Problem 1: Clearing Impurities}

In an effort to clear impurities from a fabricated nano-wire a laser beam is swept repeatedly
along the wire in the presence of a parallel electric field. After one sweep an impurity initially
at $x = 0$ has the following probability density of being found at a new position $x$

\[
p(x) = 
\begin{cases} 
\frac{1}{3} \delta(x) + \frac{2}{3} \exp(-x/a) & \text{if } 0 \leq x \leq a \\
0 & \text{elsewhere}
\end{cases}
\]
where $a$ is some characteristic length.

Give an approximate probability density for the total distance $d$ the impurity has moved 
along the wire after 36 sweeps of the laser beam.
\end{document}