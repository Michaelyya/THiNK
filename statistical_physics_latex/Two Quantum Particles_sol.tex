\documentclass{article}
\usepackage{amsmath}

\begin{document}

\title{MASSACHUSETTS INSTITUTE OF TECHNOLOGY\\
Physics Department\\
8.044 Statistical Physics I Spring Term 2013\\
Solutions to Problem Set \#2}
\maketitle

\noindent \textbf{Problem 1: Two Quantum Particles}
\begin{enumerate}
\item[(a)]
\begin{align*}
p(x_1;x_2) &= | \Psi(x_1;x_2)|^2\\
1x_1 &= (2\pi x_2)^2x_2 \exp[\frac{-x_0x_0}{1+x_2^2}]x_2^0
\end{align*}
The figure on the left shows in a simple way the location of the maxima and minima of this probability density. On the right is a plot generated by a computer application, in this case Mathematica.
\begin{itemize}
\item Max.
\item Max.$x_2$
\item $x_1$ node at $x_1$ = $x_2$
\end{itemize}
implies $p(x_1,x_2) = 0$

\item[(b)]	
\begin{align*}
p(x_1) &= \int p(x_1;x_2) dx_2 \\ 
&= \frac{1}{\sqrt{1+1x_0^2}} \exp[-\frac{x_1^2}{1+x_0^2}]\int \exp[x_2^2=\frac{x_1^2}{x_0^2}] dx_2\\ 
&= \exp [ -\frac{x_2^2}{1+x_2^2}-\frac{x_2^2}{1+x_2^2}] \\
\end{align*}
By symmetry, the result for $p(x_2)$ has the same functional form.
\begin{equation*}
p(x_1) = (\frac{x_2^2}{2\pi x_0^2}+ \frac{1}{2}) \exp[-\frac{x_1^2}{x_0^2}-\frac{x_1^2}{x_0^2}-\frac{x_1^2}{x_0^2}]
\end{equation*}
By inspection of these two results one sees that $p(x_1;x_2) =p(x_1)p(x_2)$, therefore $x_1$ and $x_2$ are not statistically independent.

\item[(c)]	
\begin{align*}
p(x_1|x_2) &= \frac{p(x_1;x_2)}{p(x_2)} \\
&= \frac{\exp[-x_2(2\pi x_0x_2)^2 \exp[ \frac{(x_1+x_2)}{x_2}=0]}{-x_2(x_2+1x_2^2-0)^2 \exp[-2\pi x_2^2=x_2^2]}\\
&= \frac{2 \pi x_1 x_2 \exp[x_1^2=x_2^0]}{-x_2^2(2\pi (1 +x_2=x_0)^2} \\
&= \frac{\exp[-x_2^2x_1^2]}{(x_2=0+1)^2 \exp[x_2^2=-x_2^2]} 
\end{align*}
It appears that these particles are anti-social: they avoid each other. For those who have had some quantum mechanics, the $\Psi(x_1;x_2)$ given here corresponds to two non-interacting spinless Fermi particles (particles which obey Fermi-Dirac statistics) in a harmonic oscillator potential.
\end{enumerate}

\noindent \textbf{Problem 2: Pyramidal Density}

\noindent \textbf{Problem 3: Stars}

\noindent \textbf{Problem 4: Kinetic Energies in an Ideal Gasses}

\noindent \textbf{Problem 5: Atomic Velocity Profile}

\noindent \textbf{Problem 6: Planetary Nebulae}
\end{document}