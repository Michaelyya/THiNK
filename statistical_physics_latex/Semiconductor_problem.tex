\documentclass[12pt]{article}
\begin{document}

\title{MASSACHUSETTS INSTITUTE OF TECHNOLOGY \\
Physics Department \\
8.044 Statistical Physics I Spring Term 2013 \\
Problem Set \#1 \\
Due in hand-in box by 12:40 PM, Wednesday, February 13}
\date{}
\maketitle
\section*{Problem 1: Doping a Semiconductor}
After diffusing impurities into a particular semiconductor the probability density $p(x)$ for finding a given impurity a distance $x$ below the surface is given by:
\[
p(x) = 
\begin{cases} 
(0.8/l) \exp(-x/l) + 0.2 \delta(x - d) & x>= 0 \\
0 & x< 0 
\end{cases}
\]
where $l$ and $d$ are parameters with the units of distance. 

\section*{Problem 2: A Peculiar Probability Density} 

Consider the following probability density. 
$p(x) = \frac{a}{b^2 + x^2}.$ 

\section*{Problem 3: Visualizing the Probability Density for a Classical Harmonic Oscillator}

Consider a particle undergoing simple harmonic motion, $x = x_0 \sin(\omega t + \phi)$, where the phase $\phi$ is completely unknown.

\section*{Problem 4: Quantized Angular Momentum}

In a certain quantum mechanical system the $x$ component of the angular momentum, $L_x$, is quantized and can take on only the three values $-I$, $0$, or $I$. 

\section*{Problem 5: A Coherent State of a Quantum Harmonic Oscillator}

In quantum mechanics, the probability density for finding a particle at a position $kr$ at time $t$ is given by the squared magnitude of the time dependent wavefunction $\Psi(kr, t):$
$p(kr, t) = |\Psi(kr, t)|^2 = \Psi^*(kr, t)\Psi(kr, t).$

\section*{Problem 6: Bose-Einstein Statistics}

You learned in 8.03 that the electro-magnetic field in a cavity can be decomposed (a 3-dimensional Fourier series) into a countably infinite number of modes, each with its own wavevector $k$ and polarization direction $\hat{k}$. 

\end{document}